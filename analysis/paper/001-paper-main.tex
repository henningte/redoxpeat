%% March 2018
%%%%%%%%%%%%%%%%%%%%%%%%%%%%%%%%%%%%%%%%%%%%%%%%%%%%%%%%%%%%%%%%%%%%%%%%%%%%
% AGUJournalTemplate.tex: this template file is for articles formatted with LaTeX
%
% This file includes commands and instructions
% given in the order necessary to produce a final output that will
% satisfy AGU requirements, including customized APA reference formatting.
%
% You may copy this file and give it your
% article name, and enter your text.
%
%
% Step 1: Set the \documentclass
%
% There are two options for article format:
%
% PLEASE USE THE DRAFT OPTION TO SUBMIT YOUR PAPERS.
% The draft option produces double spaced output.
%

%% To submit your paper:
\documentclass[draft,linenumbers]{agujournal2018}
\usepackage{apacite}
\usepackage{url} %this package should fix any errors with URLs in refs.
%%%%%%%
% As of 2018 we recommend use of the TrackChanges package to mark revisions.
% The trackchanges package adds five new LaTeX commands:
%
%  \note[editor]{The note}
%  \annote[editor]{Text to annotate}{The note}
%  \add[editor]{Text to add}
%  \remove[editor]{Text to remove}
%  \change[editor]{Text to remove}{Text to add}
%
% complete documentation is here: http://trackchanges.sourceforge.net/
%%%%%%%


%% Enter journal name below.
%% Choose from this list of Journals:
%
% JGR: Atmospheres
% JGR: Biogeosciences
% JGR: Earth Surface
% JGR: Oceans
% JGR: Planets
% JGR: Solid Earth
% JGR: Space Physics
% Global Biogeochemical Cycles
% Geophysical Research Letters
% Paleoceanography and Paleoclimatology
% Radio Science
% Reviews of Geophysics
% Tectonics
% Space Weather
% Water Resources Research
% Geochemistry, Geophysics, Geosystems
% Journal of Advances in Modeling Earth Systems (JAMES)
% Earth's Future
% Earth and Space Science
% Geohealth
%
% ie, \journalname{Water Resources Research}

\journalname{Global Biogeochemical Cycles}


\usepackage{siunitx}
\DeclareSIQualifier\carbon{C}
\DeclareSIQualifier\sample{sample}
\sisetup{
	qualifier-mode = subscript
	}
\DeclareSIUnit\wn{\cm\tothe{-1}}
\usepackage{hyperref}
\usepackage{soulutf8}
\usepackage{xr}
\externaldocument[si-]{006-paper-supplementary}
\usepackage{multibib}
\newcites{si}{References for the Supporting Information}
\usepackage{booktabs}
\usepackage{longtable}
\usepackage{array}
\usepackage{multirow}
\usepackage{wrapfig}
\usepackage{float}
\usepackage{colortbl}
\usepackage{pdflscape}
\usepackage{tabu}
\usepackage{threeparttable}
\usepackage{threeparttablex}
\usepackage[normalem]{ulem}
\usepackage{makecell}
\usepackage{xcolor}

\begin{document}

%% ------------------------------------------------------------------------ %%
%  Title
%
% (A title should be specific, informative, and brief. Use
% abbreviations only if they are defined in the abstract. Titles that
% start with general keywords then specific terms are optimized in
% searches)
%
%% ------------------------------------------------------------------------ %%

% Example: \title{This is a test title}

\title{Electrochemical Properties of Peat Particulate Organic Matter on a
Global Scale: Relation to Peat Chemistry and Degree of Decomposition}

%% ------------------------------------------------------------------------ %%
%
%  AUTHORS AND AFFILIATIONS
%
%% ------------------------------------------------------------------------ %%

% Authors are individuals who have significantly contributed to the
% research and preparation of the article. Group authors are allowed, if
% each author in the group is separately identified in an appendix.)

% List authors by first name or initial followed by last name and
% separated by commas. Use \affil{} to number affiliations, and
% \thanks{} for author notes.
% Additional author notes should be indicated with \thanks{} (for
% example, for current addresses).

% Example: \authors{A. B. Author\affil{1}\thanks{Current address, Antartica}, B. C. Author\affil{2,3}, and D. E.
% Author\affil{3,4}\thanks{Also funded by Monsanto.}}

\authors{
Henning Teickner
\affil{1}
Chuanyu Gao
\affil{1, 2}
Klaus-Holger Knorr
\affil{1}
}


% \affiliation{1}{First Affiliation}
% \affiliation{2}{Second Affiliation}
% \affiliation{3}{Third Affiliation}
% \affiliation{4}{Fourth Affiliation}

\affiliation{1}{ILÖK, Ecohydrology and Biogeochemistry Group, University of Münster,
Heisenbergstr. 2, 48149 Münster, Germany}
\affiliation{2}{Key Laboratory of Wetland Ecology and Environment, Northeast Institute
of Geography and Agroecology, Chinese Academy of Sciences, Shengbei
Street 4888, 130102, Changchun, China}
%(repeat as many times as is necessary)

%% Corresponding Author:
% Corresponding author mailing address and e-mail address:

% (include name and email addresses of the corresponding author.  More
% than one corresponding author is allowed in this LaTeX file and for
% publication; but only one corresponding author is allowed in our
% editorial system.)

% Example: \correspondingauthor{First and Last Name}{email@address.edu}
\correspondingauthor{Klaus-Holger Knorr}{klaus-holger.knorr@uni-muenster.de}
\correspondingauthor{Chuanyu Gao (Co-corresponding author)}{gaochuanyu@iga.ac.cn}

%% Keypoints, final entry on title page.

%  List up to three key points (at least one is required)
%  Key Points summarize the main points and conclusions of the article
%  Each must be 100 characters or less with no special characters or punctuation

% Example:
% \begin{keypoints}
% \item	List up to three key points (at least one is required)
% \item	Key Points summarize the main points and conclusions of the article
% \item	Each must be 100 characters or less with no special characters or punctuation
% \end{keypoints}

\begin{keypoints}
\item Peat particulate electron accepting and donating capacities are smaller
than for humic substances and dissolved organic matter
\item Both capacities are small in highly decomposed peat, unless there is a
large amount of quinones and phenols
\item We explain these patterns with parent vegetation chemistry and
conditions during decomposition
\end{keypoints}

%% ------------------------------------------------------------------------ %%
%
%  ABSTRACT
%
% A good abstract will begin with a short description of the problem
% being addressed, briefly describe the new data or analyses, then
% briefly states the main conclusion(s) and how they are supported and
% uncertainties.
%% ------------------------------------------------------------------------ %%

%% \begin{abstract} starts the second page

\begin{abstract}
Methane formation in peatlands is controlled by the availability of
electron acceptors for microbial respiration, including peat dissolved
(DOM) and particulate organic matter (POM). These electrochemical
properties are determined by peat chemistry and both are controlled by
decomposition and are thus susceptible to climatic changes or permafrost
thaw. Despite the much larger mass of POM in peat, knowledge on the
ranges of its electron transfer capacities (electron accepting capacity
-- EAC, and electron donating capacity -- EDC) is scarce in comparison
to DOM and humic substances (HS). Moreover, it is unclear how peat POM
chemistry relates to its EAC and EDC and how decomposition changes both.
To answer these questions, we compiled samples from mid to high latitude
peatlands across the world and analyzed their EAC\(_\text{POM}\) and
EDC\(_\text{POM}\), element ratios, decomposition indicators, and
relative amounts of molecular structures as derived from mid infrared
spectra. Peat EAC\(_\text{POM}\) and EDC\(_\text{POM}\) are smaller than
EAC and EDC of DOM and HS and are highly variable within and between
sites. Both are small in highly decomposed peat, unless there is a large
amount of quinones and phenols. Element ratio-based models failed to
predict EAC\(_\text{POM}\) and EDC\(_\text{POM}\), but MIRS-based models
can predict peat EAC\(_\text{POM}\) to a large extent, but not
EDC\(_\text{POM}\). We hypothesize a conceptual model that describes how
vegetation chemistry and decomposition control polymeric phenol and
quinone contents as drivers of peat electrochemical properties.
\end{abstract}
\noindent{\bf Plain language summary}\vskip-\parskip

\noindent{Peatlands accumulated large amounts of Carbon via photosynthesis and
slow decomposition of senesced plant material. Microorganisms within the
peat form methane. For this reason, peatlands are important global
sources of the greenhouse gas methane and therefore can contribute to
climate change. In order to produce methane, the microorganisms have to
transfer electrons between compounds in respiration processes. Only
recently, it has been found that the peat itself can transfer electrons
and that its capacities to accept (electron accepting capacity, EAC) and
donate (electron donating capacity, EDC) electrons are large. We
investigated which conditions favour large or low EAC and EDC of peat so
that we can better explain methane formation. We argue that vegetation
and decompsition control the amount of phenols and quinones ---
molecules in the peat that probably are responsible for most of the
peat's EAC and EDC. The EAC and EDC probably are largest for peat formed
from vegetation rich in quinones and phenols, such as shrubs, and
smaller for other vegetation types, e.g.~certain mosses. Intense
decomposition may reduce both the EAC and EDC.}
\vskip18pt
\section{Introduction}

Mid to high-latitude peatlands are important actors in the global carbon
cycle with differential effects on the global climate
\citep{Frolking.2011}. On the one side, they accumulated large amounts
of atmospheric carbon dioxide (CO\(_2\)) in the form of peat
\citep{Limpens.2008}, on the other side they are sources of methane
(CH\(_4\)) emissions to the atmosphere
\citep{Limpens.2008, Frolking.2011}.\\
Methane has 28 to 34 times the 100-year global warming potential of
CO\(_2\) and therefore may be an important driver of climate warming
\citep{Myhre.2013}. Methane emissions from peatlands increase with
higher water table depths, larger temperatures, and increased
availability of labile organic matter (OM)
\citep{Moore.1989, Yavitt.1997, Limpens.2008}. Even though net C
balances of global peatlands may be positive in the nearer future
\citep{Chaudhary.2020}, permafrost thaw and peat subsidence under warmer
temperatures may favor such conditions and hence large CH\(_4\)
emissions \citep{Anisimov.2007, Koven.2011, Frolking.2011}. For this
reason, it is important to refine our understanding of controls on
CH\(_4\) formation in peatlands.\\
Methane formation is an obligatory anaerobic process and thus occurs
only in the (at least temporally) anoxic zone of peatlands
\citep{Limpens.2008}. It can be suppressed if thermodynamically more
favorable terminal electron acceptors (TEA), such as nitrate, iron
oxides, sulfate, or organic molecules, are available
\citep{Blodau.2011, Klupfel.2014, Gao.2019}. Albeit inorganic TEAs
typically have a comparatively small abundance in peatlands, suppression
of CH\(_4\) formation can occur due to the large abundance of organic
matter (OM) that acts as TEA \citep{Gao.2019} and can be ``recharged''
even during short oxic periods
\citep{Klupfel.2014, Lau.2016, Walpen.2018b}. Whereas dissolved organic
matter (DOM) acts as electron mediator during such respiration processes
\citep{Lau.2016, Gao.2019}, particulate organic matter (POM) provides
the main part of the EAC of peat and can continuously reoxidize DOM
mediators and other TEAs \citep{Gao.2019, Blodau.2007}. Consequently,
understanding the electrochemical properties of peat POM and how these
are driven by vegetation chemistry and decomposition are key to
understand CH\(_4\) emissions from peatlands.\\
How OM chemistry controls its capacity to accept electrons (electron
accepting capacity, EAC) or donate electrons (electron donating
capacity, EDC) has been analyzed mainly for DOM
\citep{Ratasuk.2007, Aeschbacher.2010, Aeschbacher.2012, Fimmen.2007, HernandezMontoya.2012, Tan.2017, Walpen.2018, LaCroix.2020}.
The positive relation of the EAC to the C:H ratio and the EAC and EDC to
indicators for aromaticity have been attributed to quinones and phenols
representing the main contributors to the EAC and EDC, respectively
\citep{Aeschbacher.2010, Tan.2017}. In addition, sulfur and nitrogen
containing functional groups can contribute to the EAC and EDC
\citep{Fimmen.2007, Ratasuk.2007, HernandezMontoya.2012} and the
specific configuration of substituents and the degree of condensation
control the magnitude and reversibility of the redox reactions
\citep{Ratasuk.2007, Uchimiya.2009}. Overall, there is a quite detailed
conceptual understanding, which molecular structures are related to the
EAC and EDC of HS and DOM.\\
In contrast, knowledge on peat POM EAC and EDC (EAC\textsubscript{POM}
and EDC\textsubscript{POM}) and even more on how peat POM chemistry
relates to its EAC and EDC is still scarce. Few studies analyzed the EAC
and EDC of POM \citep{Keller.2013, Lau.2015, Lau.2016, Gao.2019}. These
studies focused on quantifying and analyzing the reversibility of the
electron transfer, but not how POM chemistry relates to its
electrochemical properties
\citep{Keller.2013, Lau.2015, Lau.2016, Gao.2019}. However, the ranges
of the EAC\textsubscript{POM} and EDC\textsubscript{POM} for peat POM
are not sufficiently quantified yet. Furthermore, since POM has
different chemical properties than DOM \citep{Worrall.2017}, it is
unclear if POM electrochemical properties correlate in the same way to
element ratios or molecular structures as for DOM.\\
``Classical'' indicators of peat chemistry, such as element ratios and
the nominal oxidation state of carbon (C\(_\text{OX}\)), tools such as
van-Krevelen diagrams to interpret contribution of different OM
fractions, e.g.~lignin or polysaccharides \citep{Kim.2003}, and mid
infrared spectroscopy are promising techniques to explore how peat POM
chemistry relates to its EAC and EDC. These methods have been widely
used to analyze peat chemistry and to analyze electrochemical properties
of DOM and HS
\citep{Worrall.2017, Leifeld.2012, Bader.2018, Moore.2018, Leifeld.2020, Cocozza.2003, Artz.2008, Hodgkins.2018, Tfaily.2014, Aeschbacher.2010, Aeschbacher.2012, Tan.2017, Lv.2018, LaCroix.2020}.
This would enable linking information on different OM fractions and
knowledge about assignments of molecular structures to mid infrared
variables with the EAC and EDC.\\
Similarly unexplored are the relations between the degree of
decomposition of peat and its EAC\textsubscript{POM} and
EDC\textsubscript{POM}. Since decomposition can considerably transform
peat POM and its molecular structures \citep{Cocozza.2003}, it can have
large effects on its EAC\textsubscript{POM} and EDC\textsubscript{POM}.
Developing a conceptual understanding of these relations is therefore
important to understand the mechanics of EAC\textsubscript{POM} and
EDC\textsubscript{POM} changes during decomposition. Indicators of peat
chemistry, e.g.~the N:C ratio (alternatively C:N) \citep{Biester.2014},
C\(_\text{OX}\), and MIRS-derived humification indices
\citep{Broder.2012} have been widely used to analyze the degree of
decomposition of peat \citep{Biester.2014, Drollinger.2020}.\\
A practically useful product of such investigations may be the
development of regression models to predict peat EAC\textsubscript{POM}
and EDC\textsubscript{POM} from such indicators of peat chemistry or
MIRS. For HS, a close relation between the EAC and H:C ratio was found
\citep{Aeschbacher.2010, Tan.2017}. MIRS has not been used for the
prediction of electrochemical properties of OM, but has proven useful
for its qualitative analysis \citep{HernandezMontoya.2012, Yuan.2018}
and the prediction of other peat properties
\citep{Hodgkins.2018, Artz.2008}. On the one side, such regression
models may enable to infer the EAC\textsubscript{POM} and
EDC\textsubscript{POM} of peat samples of existing data, on the other
side, they may serve at least as qualitative and (particularly for
MIRS-based models) fast and cost effective screening tools which could
assist in mapping peat EAC\textsubscript{POM} and
EDC\textsubscript{POM}.\\
To synthesize, knowledge on the relation of peat EAC\textsubscript{POM}
and EDC\textsubscript{POM} to its chemical properties and degree of
decomposition is scarce. We thus were interested in investigating how
peat chemistry relates to its EAC\textsubscript{POM} and
EDC\textsubscript{POM}, and how decomposition may change both. Moreover,
we wanted to assess if peat EAC\textsubscript{POM} and
EDC\textsubscript{POM} may be predicted from simple indicators of peat
chemistry or MIRS. To this end, we aimed to (1) quantify the
EAC\textsubscript{POM} and EDC\textsubscript{POM} of peat material
formed under various environmental conditions and with a range of
degrees of decomposition under standardized conditions, (2) analyze the
relation of the EAC\textsubscript{POM} and EDC\textsubscript{POM} to
bulk chemical properties, indicators for decomposition, and molecular
structures, and (3) evaluate if element ratios (H:C, O:C, N:C, S:C) and
MIRS can be used to predict peat EAC\textsubscript{POM} and
EDC\textsubscript{POM}, all based on a global data set of peat
materials.\\
These analyses represent a basis for the conceptual understanding of how
peat chemistry and decomposition affect the EAC\textsubscript{POM} and
EDC\textsubscript{POM}. This knowledge can help understanding,
quantifying, and predicting spatiotemporal variability in peat
EAC\textsubscript{POM} and EDC\textsubscript{POM} and thus contribute to
the quantitative understanding of peatland decomposition processes and
CH\(_4\) formation on a global scale. With this, we aim to contribute
information to better understand and predict peatland-climate
interactions.

\section{Materials and methods}

To answer our main research question --- how peat chemistry relates to
its EAC and EDC, and how decomposition may change both --- we compiled
peat material and data for peatland sites with a broad range of climate
regimes, dominant vegetation, and peatland types and performed various
measurements under standardized conditions to describe their chemical
and electrochemical characteristics. With this data, we conducted an
explorative analysis, computed regression models that predict the
EAC\textsubscript{POM} and EDC\textsubscript{POM} from element ratios or
MIRS, and finally developed a conceptual model how initial peat
chemistry and decomposition change chemical characteristics and the
EAC\textsubscript{POM} and EDC\textsubscript{POM} of peat.

\subsection{Study sites}

We compiled peat cores collected in the course of different projects
from 15 peatland sites (one core per site). The different peatland sites
are spread across the range of mid to high-latitude peatland areas
(figure \ref{fig:study-sites-map-out}), experience different climatic
regimes, and comprise both bogs and fens with different dominant
vegetation cover (table \ref{tab:t-study-sites}). This enabled us to
analyze peat material with a representative range of chemical properties
and EAC and EDC.\\
Peat samples were selected from four approximate depth levels, depending
on the vertical resolution of the peat cores and the maximum depth
reached during coring (10 to 20, 30 to 40, 60 to 70 cm, and the deepest
available sample per core at depths of 140 to 320 cm). We assumed that
the samples cover most of the decomposition and vegetation-shift related
variability in peat chemistry and hence EAC and EDC.

\begin{figure}[H]

{\centering \includegraphics[width=0.6\textwidth]{001-paper-main_files/figure-latex/study-sites-map-out-1} 

}

\caption{Map of the peatland sites from which peat material and data were compiled for this study. The map was created using data from the R package rnaturalearth \citep{South.2017}.}\label{fig:study-sites-map-out}
\end{figure}

\begin{landscape}\begin{table}

\caption{\label{tab:t-study-sites}Overview on the sites from which peat samples and data were derived from. Longitude and latitude are given in decimal degree North and East, respectively, the altitude is given in meter above sea level, peatland type differentiates between bogs and fens following available studies for the respective sites or from own investigations following concepts in \citet{Rydin.2013}. "Temperature" is the mean annual temperature [$^{\circ}$C], "Precipitation" is the total annual precipitation [mm], and "References" are references with additional information on the sites. Elevation data were derived from the median values of the GMTED2010 data \citep{Danielson.2011}. All climate data were derived from the WorldClim version 2.1 climate data for 1970-2000 (30 seconds spatial resolution, monthly temporal resolution) \citep{Fick.2017}.}
\centering
\def\arraystretch{0.6}
\resizebox{\linewidth}{!}{
\begin{tabular}[t]{clcccc>{\raggedright\arraybackslash}p{4cm}cc>{\raggedright\arraybackslash}p{4cm}}
\toprule
Site label & Site name & Longitude & Latitude & Altitude & Peatland Type & Current vegetation & Temperature & Precipitation & References\\
\midrule
BB & Beerberg & 10.74 & 50.66 & 977 & Bog & \emph{Sphagnum}, shrubs & 5.3 & 1349 & \\
MK & Martinskapelle & 8.15 & 46.10 & 2089 & Bog & Shrubs, \emph{Sphagnum} & 2.1 & 1027 & \\
LT & La Tenine & 6.93 & 48.04 & 863 & Bog & \emph{Sphagnum}, shrubs & 6.4 & 1330 & \\
DE & Degerö & 19.56 & 64.18 & 275 & Fen & \emph{Sphagnum}, sedges, shrubs & 1.7 & 621 & \citet{Sagerfors.2008}\\
ISH & Ishimbaevskoye & 65.34 & 57.47 & 77 & Fen & Shrubs, \emph{Sphagnum} & 1.5 & 472 & \citet{Wertebach.2016}\\
\addlinespace
KR & Kyzyltun Ryam & 69.62 & 56.26 & 110 & Bog & \emph{Sphagnum} & 1.0 & 384 & \citet{Larina.2013}\\
TX & Touxi & 127.84 & 42.28 & 1070 & Fen & Vascular plants, \emph{Sphagnum} & 1.3 & 754 & \\
DT & Dongtu & 127.86 & 42.27 & 1268 & Fen & Vascular plants, \emph{Sphagnum} & 0.6 & 775 & \\
LB & Lutose Bog & -117.17 & 59.48 & 309 & Bog & \emph{Sphagnum}, shrubs & -1.8 & 356 & \citet{Heffernan.2020}\\
LP & Lutose Plateau & -117.17 & 59.48 & 309 & Bog & \emph{Sphagnum}, lichens & -1.8 & 356 & \citet{Heffernan.2020}\\
\addlinespace
MB & Mer Bleue & -75.52 & 45.41 & 68 & Bog & Shrubs, \emph{Sphagnum} & 5.6 & 945 & \citet{Elliott.2012}\\
PBR & P. Brunswick & -70.97 & -53.64 & 50 & Bog & \emph{Sphagnum}, shrubs & 6.0 & 797 & \citet{Broder.2012}\\
SKY I-1 & Skyring I-1 & -72.45 & -52.14 & 75 & Bog & Vascular plants (\emph{Astelia pumila}), \emph{Sphagnum} & 6.1 & 637 & \citet{Mathijssen.2019}\\
SKY I-6 & Skyring I-6 & -72.45 & -52.14 & 75 & Bog & \emph{Sphagnum} & 6.1 & 637 & \citet{Mathijssen.2019}\\
SKY II & Skyring II & -72.13 & -52.51 & 36 & Bog & \emph{Sphagnum} & 6.3 & 690 & \citet{Broder.2012}\\
\bottomrule
\end{tabular}}
\end{table}
\end{landscape}

\subsection{EAC and EDC measurements}

We measured the EAC and EDC of the peat samples using mediated
electrochemical reduction (MER) and oxidation (MEO), respectively,
following largely the protocols provided by \citet{Lau.2015} and
\citet{Gao.2019}.\\
For this, the peat material was freeze dried (alpha 1-4 plus, Christ,
Osterode, Germany) and finely ground to powder in a vibratory cup mill
(tungsten carbide cups; Retsch MM 400, Haan, Germany). The ground
samples were suspended in water in order to create a suspension that can
be pipetted for analyses as described elsewhere \citep{Lau.2015}. For
this, approximately \(\SI{0.08}{\g}\) of sample and
\(\SI{30}{\milli\L}\) of deionized, degassed, and anoxic water were
used, or in case of lower amounts of sample available a similar ratio of
water to solids.\\
The suspensions were transferred into a glove box with N\(_2\)
atmosphere (\(<\SI{1}{ppm}\) O\(_2\); Inert Lab Glovebox, Innovative
Technology, Amesbury, MA, USA) to perform the electrochemical
measurements. For each measurement, an aliquot of each suspension,
depending on the total organic carbon content and the expected range of
the EAC/EDC (typically \(\SI{100}{\micro\L}\) suspension containing
\(\SI[separate-uncertainty=true, multi-part-units = single]{0.2 \pm 0.01}{\milli\g}\)),
was transferred into electrochemical cells. The suspensions were
continuously stirred (topolino, IKA, Staufen, Germany) to ensure
reproducible transfer into the electrochemical cells.\\
The electrochemcial cells and analytical setup consisted of a
multichannel potentiostat (CH1000, CH Instruments, Austin, TX, USA),
glassy carbon working electrodes (Sigradur, HTW, Thierhaupten, Germany),
platinum counter electrodes (coiled \(\SI{0.4}{\milli\m}\) platinum
wire, Sigma-Aldrich, St.~Louis, USA), and Ag/AgCl reference electrodes
(RE-1B, ALS Co.~Ltd, Tokyo, Japan). All potentials were experimentally
measured against Ag/AgCl reference, but are reported versus the standard
hydrogen electrode.\\
The working electrode solution contained KCl as a background electrolyte
(\(\SI{0.1}{\mol\per\L}\)) and was buffered to pH 7
(\(\SI{0.2}{\mol\per\L}\) KH\(_2\)PO\(_4\)) to ensure stable pH during
measurements \citep{Aeschbacher.2011} and to enable direct comparisons
with available data \citep{Aeschbacher.2010, Walpen.2018, Tan.2017}.
Prior to analyses of samples \(\SI{180}{\micro\L}\) of a
\SI{0.1}{\mol\per\L} solution of the mediator diquat
(6,7--dihydrodipyrido {[}1,2--a:20,10--c{]} pyraziniumdibromid
monohydrate; EH\(_0=\SI{-0.36}{\V}\); Supelco, USA; 95\% purity) was
added for MER, and a similar amount of ABTS
(2,2--azino--bis-(3--ethylbenzthiazoline--6--sulfonic acid) ammonium
salt; EH\(_0=\SI{+0.68}{\V}\); Sigma Aldrich, St.~Louis, USA; 98\%
purity) for MEO.\\
Values of EAC and EDC were determined in MER and MEO, respectively, by
integrating the reductive or oxidative current signals over time and
normalizing the quantified numbers of electrons transferred to the
amount of carbon added for analysis \citep{Aeschbacher.2010}. The sum of
EAC and EDC is referred to as total electron exchange capacity
(EEC\textsubscript{tot}).\\
Strictly, the obtained EAC and EDC values are the combined EAC and EDC
of the POM, DOM, and dissolved inorganic ions and inorganic particles
that could be reduced and oxidized, respectively. Prior studies of
organic rich sediments and peat samples suggest that dissolved inorganic
ions and DOM have a negligible contribution to the EAC and EDC of bulk
peat material (\textasciitilde1\%) \citep{Lau.2015, Gao.2019} and
therefore we assume that our measurements are representative for the
solid phase.\\
Even though POM is assumed to be the dominant contributor to the EAC and
EDC of organic rich sediments such as peat \citep{Lau.2015, Gao.2019},
solid iron phases can contribute to the EAC and EDC of peat as measured
by MER/MEO \citep{Lau.2015}. We therefore corrected the measured EAC and
EDC values for contributions of Fe\(^{2+}\) (each mol contributing one
mol electrons to the EDC) and Fe\(^{3+}\) (each mol contributing one mol
electrons to the EAC) \citep{Lau.2015, Gao.2019}. To this end, we
extracted iron by adding \SI{4}{\milli\liter} \SI{1}{\mol\per\L} HCl to
\SI{1}{\milli\liter} of each sample (in some cases less material had to
be used), letting the suspensions rest for \SI{72}{\hour} in the dark,
and filtering the solution through \SI{0.22}{\micro\meter} Nylon syringe
filters. Concentrations of Fe\(^{2+}\) and Fe\(^{3+}\) in the filtrate
were measured spectrophotometrically using the 1,10--phenanthroline
method \citep{Tamura.1974} and from this, we computed the contributions
of iron to the EAC (EAC\textsubscript{Fe$^{3+}$}) and EDC
(EDC\textsubscript{Fe$^{2+}$}), respectively. The EAC\textsubscript{POM}
was then computed by subtracting EAC\textsubscript{Fe$^{3+}$} from the
measured EAC and the EDC\textsubscript{POM} by subtracting
EDC\textsubscript{Fe$^{2+}$} from the measured EDC \citep{Lau.2015}.\\
There are several known issues with this procedure. First, acid
extraction may not extract all redox active iron moieties
\citep{Lau.2016}. Second, during acid extraction, the redox equilibrium
between iron and OM is shifted and Fe\(^{3+}\) may in part be reduced to
Fe\(^{2+}\) \citep{Lau.2015}. Consequently, the contribution of iron to
the EDC is typically overestimated, whereas the contribution of iron to
the EAC is typically underestimated \citep{Lau.2015}.\\
The first issue may be negligible for most samples because peat
typically contains few mineral particles and hence most iron typically
is acid extractable (supporting figure
\ref{si-fig:p-fe-xrf-comparison-res}). However, we cannot fully exclude
that more iron moieties than that accessible via acid extraction
contributed to the EAC and EDC for samples with larger iron contents
(supporting figure \ref{si-fig:p-fe-xrf-comparison-res}). The third
issue probably affected our calculated EAC\textsubscript{POM} and
EDC\textsubscript{POM} values. One indication for this is that 5
EDC\textsubscript{POM} values were negative (minimum:
\SI{-8}{\micro\mol\per\gram\carbon}). We therefore assume that the
calculated EAC\textsubscript{POM} and EDC\textsubscript{POM} values are
biased for samples with high iron content and considered this during
data analysis and interpretation.\\
We finally report the EAC\textsubscript{POM} and EDC\textsubscript{POM}
relative to the C mass of the measured sample (mass of C in the POM
suspension aliquot). Moreover, we computed the EAC\textsubscript{POM}
and EDC\textsubscript{POM} relative to the total mass of the sample for
comparison with values reported in other studies. For each sample, we
computed average EAC\textsubscript{POM} and EDC\textsubscript{POM}
values and respective standard deviations from the replicate
measurements. During this, we discarded one EAC replicate measurement
for which we assumed measurement errors because it differed extremely
(more than \(\SI{1000}{\micro\mol\per\gram\carbon}\)) from the remaining
replicate measurements for the same sample (supporting figure
\ref{si-fig:el-preprocessing-p1-res}).

\subsection{Element contents}

We analyzed concentrations of C, N, and S for all samples by catalytic
combustion using an elemental analyzer (EA 3000, Eurovector, Pavia,
Italy). Concentrations of H and O were determined based on the modified
Dumas Method, using an CHNS/O analyzer (FlashEA 1112, Thermo Fisher
Scientific, Delft, The Netherlands). The nominal oxidation state of C
(C\(_\textrm{ox}\)) was computed from the contents of C, H, N, and O
\citep{Masiello.2008, Worrall.2016b}.\\
Total concentrations of other elements (Fe, P and others; see supporting
table \ref{si-tab:t-cor-el-xrf} for a full list of measured elements)
were determined by wavelength dispersive X-ray fluorescence spectroscopy
(WD-XRF; ZSX Primus II, Rigaku, Tokyo, Japan) calibrated with a set of
15 reference materials, consisting of certified plant, peat, and
sediment materials, and 5 in-house working standards. Analyses were done
on \(\SI{500}{\milli\g}\) of ground, powdered sample, pressed to a
\(\SI{13}{\milli\m}\) pellet (without pelleting aids) at a load of
approximately \SI{7}{\tonne}. For few samples, S contents were derived
from the WD-XRF data.

\subsection{Mid infrared spectroscopy}

Fourier-transform mid infrared spectra (MIRS) were used to obtain
detailed information on peat molecular structures and to compute
regression models for the prediction of peat EAC and EDC. Two mg of
powdered sample were mixed with 200 mg KBr (FTIR grade, Sigma Aldrich,
St.~Louis, MO, USA) and pressed to a 13 mm pellet. Spectra were recorded
on a Cary 660 FTIR spectrometer (Agilent, Santa Clara, CA, USA) in the
range 650 to 4000 cm\(^{-1}\) at a resolution of 0.5 to \SI{2}{\wn}. A
number of 32 scans per sample were collected in the absorbance mode and
a KBr background was subtracted.\\
The recorded MIRS were preprocessed to remove known artifacts and
harmonize the data. Spectral preprocessing was performed using the
package ir (0.0.0.9000) \citep{Teickner.2020}. To assess the degree of
peat decomposition, we computed a humification index by dividing the
intensity at \(\SI{1630}{\wn}\) and \(\SI{1090}{\wn}\) (denoted as
HI\textsubscript{1630/1090}) \citep{Broder.2012} using irpeat
(0.0.0.9000) \citep{Teickner.2020b}.\\
Two transformed versions of the spectra were created for the computation
of MIRS-based regression models: The first is a binned version and the
second is a derived and binned version of the preprocessed spectra.
Binning was performed with a bin width of \SI{10}{\wn} to reduce
autocorrelation and noise in the spectra. Prior to binning, the
preprocessed spectra were derived using a Savitzgy-Golay filter (filter
width: \SI{5}{\wn}) \citep{signaldevelopers.2014}. Derivatization can
improve the resolution of features and therefore can improve the
predictive accuracy of regression models
\citep{Stuart.2005, Engel.2013}.

\subsection{Statistical analyses}

\subsubsection{\texorpdfstring{Balancing the analytical bias in
EAC\textsubscript{POM} and EDC\textsubscript{POM}
values}{Balancing the analytical bias in EAC and EDC values}}

As mentioned in the previous section, the EAC\textsubscript{POM} and
EDC\textsubscript{POM} values computed from measured EAC and EDC values
and acid extracted iron ion contents probably are biased, especially for
samples with high iron content \citep{Lau.2015, Lau.2016}. Since we are
interested in electrochemical properties of POM, this bias must be
considered during all data analysis steps. We aimed to do this first by
quantifying the maximum potential contribution of iron to the EAC and
EDC, second, by creating a filtered data set for each the
EAC\textsubscript{POM} and EDC\textsubscript{POM} containing only data
where the maximum potential contribution of iron to the EAC and EDC was
\(\le\)\SI{100}{\micro\mol\per\gram\carbon}, respectively, and third,
performing all analyses for the filtered data set. Analyses for the
unfiltered data set can be found in the supplementary information.\\
The maximum potential contribution of iron to the EAC and EDC was
determined as the total concentration of iron in the acid extract (sum
of the concentrations of Fe\(^{2+}\) and Fe\(^{3+}\)). We assumed that
non-extracted iron is redox inactive and hence this total iron
concentration reflects the maximum potential contribution of iron to
either the EAC or EDC, irrespective of the initial redox state and redox
state changes during the acid extraction. This assumption is true for
samples with low total iron contents, but cannot be validated for
samples with larger iron contents (supporting figure
\ref{si-fig:p-fe-xrf-comparison-res}) \citep{Lau.2016}. An overview on
the maximum potential contribution of iron to the EAC and EDC across all
samples computed from the acid extractable iron content can be found in
supporting figure \ref{si-fig:p-fe-contribution-abs-res}. The threshold
of \(\le\)\SI{100}{\micro\mol\per\gram\carbon} was chosen to balance the
reduction in the analytical bias in the filtered data and the reduction
in sample size, resulting in sample sizes of 52 for the EAC and EDC. As
a result, there was only a constant offset bias between corrected and
uncorrected values for the EAC and EDC (supporting figure
\ref{si-fig:p-fe-corrected-uncorrected-res}).\\
In general, measured EAC and EDC values typically are much larger than
\SI{100}{\micro\mol\per\gram\carbon} and samples with large acid
extracted (and total) iron contents typically have a small EAC and EDC
(supporting figures \ref{si-fig:p-fe-corrected-uncorrected-res} and
\ref{si-fig:p-fe-raw-iron}), suggesting that the maximum bias is
probably low. Therefore we assume that it is unlikely that the remaining
bias had a large influence on the results. As mentioned above, we
cannot, however, fully exclude contributions of non-acid extractable
iron minerals to the EAC and EDC. We also note that choosing a tighter
filter threshold is likely to cause selection bias since samples with
high acid extracted iron content tend to be more decomposed (supporting
figure \ref{si-fig:p-fe-decomposition-res}).

\subsubsection{\texorpdfstring{EAC\textsubscript{POM} and
EDC\textsubscript{POM} variability}{EAC and EDC variability}}

We created several plots and computed Pearson correlation coefficients
to analyze patterns in the samples' EAC\textsubscript{POM} and
EDC\textsubscript{POM}. In particular, we compared our
EAC\textsubscript{POM} and EDC\textsubscript{POM} values with those
measured for various HS and DOM samples in other studies
\citep{Aeschbacher.2012, Tan.2017, Walpen.2018}. These values from other
studies were extracted from the publications' figures using the R
package digitize \citep{Poisot.2011}.

\subsubsection{Relation to Chemical Properties, Decomposition
Indicators, and Infrared Spectra}

To analyze the relation of the EAC\textsubscript{POM},
EDC\textsubscript{POM}, and
EDC\textsubscript{POM}:EAC\textsubscript{POM} ratio to different
indicators of bulk peat chemistry (H:C, O:C, N:C, S:C ratio,
HI\(_\text{1630/1090}\), and C\(_\text{OX}\)), we created scatterplots
and computed their pairwise Pearson correlation (\(\rho\)). The H:C
ratio has been shown to relate negatively to the EAC and EDC of HS
\citep{Aeschbacher.2010, Tan.2017, Lv.2018}. The O:C, N:C, S:C ratio,
and C\(_\text{OX}\) are indicators for peat decomposition
\citep{Masiello.2008, Biester.2014, Leifeld.2012} and C\(_\text{OX}\)
was positively related to the EDC of HS \citep{Lv.2018}. The O:C ratio
is an indicator for the amount of polysaccharides \citep{Kim.2003} which
are not assumed to contribute to the EAC\textsubscript{POM} and
EDC\textsubscript{POM}. There is some evidence for the contribution of
nitrogen and sulfur containing functional groups to the
EAC\textsubscript{POM} and EDC\textsubscript{POM}
\citep{Ratasuk.2007, Fimmen.2007, HernandezMontoya.2012} and the N:C,
and S:C ratio might give information on this, too.\\
To qualitatively analyze potential joint effects of the H:C and O:C
ratio and relate their variability to different OM fractions, we created
van-Krevelen plots. For some International Humic Substances Society
(IHSS) reference samples, information on element contents
\citep{HuffmanLaboratories.NA} and EAC and EDC \citep{Aeschbacher.2012}
are available. We therefore could include these samples in the
van-Krevelen plots. Finally, we computed correlation spectra for the
EAC\textsubscript{POM} and EDC\textsubscript{POM} by computing the
Pearson correlation of both variables with each MIRS variable. This
allowed us to investigate relations to molecular structures derived from
MIRS.

\subsubsection{Regression Models}

We used regression models to analyze if peat EAC\textsubscript{POM} and
EDC\textsubscript{POM} can be predicted from a linear combination of
individual element ratios (H:C, O:C, N:C, S:C). Different regression
approaches were investigated: (1) including only the H:C and O:C ratio,
and (2) including all four element ratios. We used Bayesian hierarchical
models to consider the uncertainties of the replicate measurements and
that these were different for different samples (supporting figure
\ref{si-fig:p-reg-distogram-res} provides an overview on the model
structure). For the EAC\textsubscript{POM} and EDC\textsubscript{POM}
values (both individual measurements and replicate measurement averages)
we assumed a Gamma distribution with log-link function. Individual
replicate measurements with EDC\textsubscript{POM} values
\(\le \SI{0}{\micro\mol\per\gram\carbon}\) were set to
\SI{0}{\micro\mol\per\gram\carbon} for this. Since the absolute values
of these measurements is small in comparison to the median
EDC\textsubscript{POM} values, we assumed that this causes no bias in
our analyses. Further details on the specification of the element
ratio-based models can be found in the supporting information.\\
Additionally, we computed regression models to analyze the joint
relation of molecular structures derived from MIRS to the
EAC\textsubscript{POM} and EDC\textsubscript{POM}. For this, we used
partial least squares regression (PLSR) and Bayesian regularization
\citep[e.g.~][]{Artz.2008, Ferragina.2015}. Since different
preprocessing approaches for MIRS can affect the predictive performance
of models \citep{Engel.2013}, we computed not only one regression model
for each the EAC\textsubscript{POM} and EDC\textsubscript{POM}, but
four, using either non-derived spectra or first derivative spectra as
input data, and PLSR or Bayesian regularization, respectively. In
addition, for the EDC\textsubscript{POM}, we observed biases and tried
to reduce these by dividing the EDC\textsubscript{POM} by
HI\textsubscript{1630/1090}.\\
We applied Bayesian projection \citep{Piironen.2020} on all MIRS-based
models to facilitate their interpretation. Bayesian projection seeks to
find a reduced model that is similar to the full model in terms of its
posterior distribution within some defined tolerance threshold
\citep{Piironen.2020}. This allowed us to establish more direct links
between the underlying molecular structures and the
EAC\textsubscript{POM} and EDC\textsubscript{POM}. Further details on
the projection approach are described in the supporting information.\\
We also wanted to analyze if regression models using element ratios or
mid infrared variables may be used to predict peat
EAC\textsubscript{POM} and EDC\textsubscript{POM}. For this, we
estimated the predictive accuracy using 10-fold cross-validation (CV)
\citep{Roberts.2017} and the root mean square error (RMSE) as
performance metric \citep{Bellocchi.2010}. As we did not observe large
deviations between alternative regression models, we cross-validated
only the regression models containing all element ratios (H:C, O:C, N:C,
S:C) and the MIRS-based models using Bayesian regularization and
non-derived spectra. The RMSE was computed between the MCMC draws for
predictions of the regression models for new observations and MCMC draws
for estimates of the samples' average EAC\textsubscript{POM} or
EDC\textsubscript{POM}, respectively. This allowed us to consider the
uncertainties of the repeated measurements during CV, and to compute a
probability distribution of the RMSE values based on the CV uncertainty,
as well as the models' uncertainties.

\section{Results}

\subsection{\texorpdfstring{EAC\textsubscript{POM} and
EDC\textsubscript{POM}}{EAC and EDC}}

Our peat samples' average EAC\textsubscript{POM} ranges between 179 and
\SI{1228}{\micro\mol\per\g\carbon} and their EDC\textsubscript{POM}
between 9 and \SI{569}{\micro\mol\per\g\carbon} (after filtering; see
table \ref{tab:t-ranges-sites} for site-specific values). The
EAC\textsubscript{POM} is typically larger than the
EDC\textsubscript{POM}, with few exceptions, and both are loosely
positively related (\(\rho=\) 0.39), as shown in figure
\ref{fig:p-el-comparison-res} b.\\
The EAC\textsubscript{POM} and EDC\textsubscript{POM} vary considerably
between different sites and the depth-related within-site variability is
also large (figure \ref{fig:p-el-comparison-res} a). In addition, no
general depth-related pattern is visible: For some peat cores, the
uppermost sample had the largest EAC\textsubscript{POM} or
EDC\textsubscript{POM}, for others the deepest, and for some the
intermediate samples. The same figure also shows that the
EAC\textsubscript{POM} and EDC\textsubscript{POM} can change
independently or even inversely to each other along the peat profile.
Finally, our specific sites did not show distinct patterns in their
EAC\textsubscript{POM} and EDC\textsubscript{POM} to differentiate bogs
and fens (supporting figure
\ref{si-fig:el-comparison-peatland-types-res}).\\
Figure \ref{fig:p-el-comparison-res} C compares EAC\textsubscript{POM}
and EDC\textsubscript{POM} for our peat samples to those for different
HS and peat DOM samples obtained from different studies
\citep{Aeschbacher.2012, Tan.2017, Walpen.2018}. Most HS and peat DOM
samples have a clearly larger EAC and EDC, whereby mineral soil HS
\citep{Tan.2017} seem to have intermediate values relative to our
measurements on the one side and the larger values for various IHSS
reference samples \citep{Aeschbacher.2012} and peat DOM samples
\citep{Walpen.2018} on the other side.

\begin{landscape}\begin{table}

\caption{\label{tab:t-ranges-sites}Overview on the electrochemical and chemical properties and decomposition states of the peat samples for the different sites (mean, min, max). EAC\textsubscript{POM} and EDC\textsubscript{POM} are given in $\si{\micro\mol\per\g\carbon}$. Element ratios are molar ratios. C$_\text{OX}$ is the nominal oxidation state of carbon. HI$_\text{1630/1090}$ is the ratio of the intensities at 1630 and $\SI{1090}{\wn}$ computed from mid infrared spectra.}
\centering
\resizebox{\linewidth}{!}{
\begin{tabular}[t]{lcccccccl}
\toprule
Site label & EAC$_\text{POM}$ & EDC$_\text{POM}$ & H:C & O:C & N:C & S:C & C$_\text{OX}$ & HI$_\text{1630/1090}$\\
\midrule
BB & 477 (390, 534) & 397 (327, 463) & 1.46 (1.4, 1.5) & 0.68 (0.62, 0.74) & 0.01 (0.01, 0.02) & 0.001 (0, 0.003) & -0.07 (-0.12, 0.02) & 0.56 (0.44, 0.62)\\
MK & 581 (339, 940) & 341 (213, 492) & 1.44 (1.43, 1.45) & 0.61 (0.52, 0.69) & 0.02 (0.01, 0.03) & 0.001 (0, 0.001) & -0.16 (-0.29, -0.01) & 0.72 (0.55, 0.86)\\
LT & 542 (427, 734) & 310 (161, 452) & 1.49 (1.48, 1.52) & 0.68 (0.66, 0.7) & 0.01 (0.01, 0.02) & 0.001 (0, 0.001) & -0.1 (-0.13, -0.05) & 0.58 (0.5, 0.63)\\
DE & 477 (424, 590) & 466 (288, 673) & 1.55 (1.49, 1.61) & 0.66 (0.6, 0.7) & 0.02 (0.01, 0.02) & 0 (0, 0.001) & -0.18 (-0.25, -0.08) & 0.54 (0.49, 0.59)\\
ISH & 633 (435, 926) & 227 (176, 349) & 1.42 (1.35, 1.48) & 0.57 (0.54, 0.61) & 0.02 (0.02, 0.03) & 0.001 (0.001, 0.003) & -0.22 (-0.28, -0.13) & 0.93 (0.66, 1.1)\\
\addlinespace
KR & 721 (486, 907) & 402 (276, 548) & 1.5 (1.39, 1.56) & 0.65 (0.59, 0.69) & 0.02 (0.01, 0.02) & 0.001 (0.001, 0.001) & -0.16 (-0.25, -0.09) & 0.75 (0.64, 1.01)\\
TX & 415 (302, 636) & 80 (36, 128) & 1.51 (1.47, 1.56) & 0.56 (0.53, 0.61) & 0.04 (0.04, 0.05) & 0.002 (0.001, 0.002) & -0.27 (-0.31, -0.2) & 0.83 (0.61, 0.98)\\
DT & 325 (255, 388) & 121 (98, 140) & 1.47 (1.4, 1.53) & 0.5 (0.48, 0.53) & 0.04 (0.04, 0.05) & 0.002 (0.001, 0.002) & -0.33 (-0.38, -0.31) & 0.89 (0.8, 0.99)\\
LB & 642 (324, 1208) & 363 (284, 410) & 1.48 (1.29, 1.63) & 0.68 (0.57, 0.73) & 0.02 (0.01, 0.02) & 0.001 (0, 0.002) & -0.06 (-0.13, -0.01) & 0.45 (0.37, 0.62)\\
LP & 968 (691, 1228) & 269 (207, 334) & 1.4 (1.28, 1.54) & 0.65 (0.6, 0.7) & 0.02 (0.01, 0.02) & 0.001 (0, 0.001) & -0.04 (-0.08, 0) & 0.61 (0.55, 0.66)\\
\addlinespace
MB & 758 (656, 868) & 351 (247, 569) & 1.41 (1.31, 1.46) & 0.64 (0.52, 0.72) & 0.02 (0.01, 0.03) & 0.001 (0, 0.001) & -0.09 (-0.21, 0.05) & 0.69 (0.47, 0.95)\\
PBR & 799 (455, 1192) & 461 (376, 541) & 1.38 (1.3, 1.45) & 0.6 (0.56, 0.65) & 0.02 (0.01, 0.02) & 0.001 (0.001, 0.001) & -0.12 (-0.15, -0.11) & 0.83 (0.7, 0.95)\\
SKY I-1 & 323 (179, 468) & 76 (10, 155) & 1.43 (1.38, 1.51) & 0.58 (0.55, 0.63) & 0.03 (0.02, 0.04) & 0 (0, 0.001) & -0.2 (-0.29, -0.07) & 0.9 (0.54, 1.13)\\
SKY I-6 & 346 (234, 497) & 169 (58, 269) & 1.44 (1.4, 1.48) & 0.55 (0.48, 0.62) & 0.02 (0.02, 0.02) & 0 (0, 0.001) & -0.28 (-0.38, -0.14) & 0.92 (0.65, 1.37)\\
SKY II & 749 (570, 912) & 376 (347, 403) & 1.46 (1.44, 1.48) & 0.67 (0.53, 0.74) & 0.01 (0.01, 0.02) & 0 (0, 0.001) & -0.07 (-0.32, 0.08) & 0.58 (0.44, 0.73)\\
\bottomrule
\end{tabular}}
\end{table}
\end{landscape}

\begin{figure}[H]

{\centering \includegraphics[width=0.7\textwidth]{001-paper-main_files/figure-latex/p-el-comparison-res-1} 

}

\caption{\textbf{A}: Depth profiles of the EAC\textsubscript{POM} and EDC\textsubscript{POM} for each peat core. Thick curves represent the median values of the replicate measurements and horizintal lines the standard error of the replicate measurements. Sites are sorted according to their median EAC\textsubscript{POM}. The depth is presented log-scaled. Different colours indicate samples for which the potential maximum contribution of iron to the EAC or EDC is larger than $\SI{100}{\micro\mol\per\gram\carbon}$. \textbf{B}: Plot of the average EAC\textsubscript{POM} versus the average EDC\textsubscript{POM}. Error bars represent the respective standard errors from replicate measurements. Samples below the diagonal line have a larger EAC\textsubscript{POM} than EDC\textsubscript{POM}. Only samples with a potential contribution of iron to the EAC and EDC $\le\SI{100}{\micro\mol\per\gram\carbon}$ are shown. \textbf{C}: Histograms of the EAC and EDC values, respectively, for the peat POM samples analyzed in this study, humic substances, and dissolved organic matter analysed in other studies. Top: Values reported in $\si{\micro\mol\per\g\sample}$. Bottom: Values reported in $\si{\micro\mol\per\g\carbon}$). Only samples with a potential contribution of iron to the EAC and EDC $\le\SI{100}{\micro\mol\per\gram\carbon}$ are shown.}\label{fig:p-el-comparison-res}
\end{figure}

\subsection{Relation to Bulk Chemical Properties}

No single bulk chemical property is strongly related to either the
EAC\textsubscript{POM}, EDC\textsubscript{POM} or
EDC\textsubscript{POM}:EAC\textsubscript{POM} ratio, except for the N:C
ratio which is relatively strongly related to the EDC\textsubscript{POM}
(\(\rho=\) -0.65), and the H:C ratio which is related to the
EAC\textsubscript{POM} (\(\rho=\) -0.47) (supporting figure
\ref{si-fig:p-el-variables1-res}). The
EDC\textsubscript{POM}:EAC\textsubscript{POM} ratio is related to the
N:C ratio (\(\rho=\) -0.5). All other Pearson correlation coefficients
between either the EAC\textsubscript{POM} or EDC\textsubscript{POM} and
any of the variables we considered is not larger than 0.45 or smaller
than -0.43.\\
The relative weak correlations are not due to non-linear relations, but
due to a large variability within and between sites (supporting figure
\ref{si-fig:p-el-variables1-res}). For the EAC\textsubscript{POM}, this
variability is especially large for samples with average N:C ratio,
medium to large O:C ratio, large C\(_\text{OX}\), and medium
HI\(_\text{1630/1090}\). For the EDC\textsubscript{POM}, the variability
is large for medium to small O:C ratio, medium to small C\(_\text{OX}\),
and medium HI\(_\text{1630/1090}\). For the
EDC\textsubscript{POM}:EAC\textsubscript{POM} ratio, the variability is
large for small to medium N:C ratios and HI\(_\text{1630/1090}\).\\
EAC\textsubscript{POM} and EDC\textsubscript{POM} values are large along
a gradient with large O:C and large H:C ratio at the one end and small
O:C and small H:C ratio at the other end (figure
\ref{fig:p-el-ch-co1-res}). Conversely, samples with small O:C ratio,
but large H:C ratio have the smallest EAC\textsubscript{POM} and
EDC\textsubscript{POM}. Whilst this gradient is a general pattern, there
are some samples with smaller and larger EAC\textsubscript{POM} and
EDC\textsubscript{POM} values than could be expected based on the
described gradient. For instance the two samples from SKY I-1 at
intermediate depth have a surprisingly small EAC\textsubscript{POM} and
EDC\textsubscript{POM}, and samples from SKY II and Martinskapelle (MK)
have a large EAC\textsubscript{POM} and EDC\textsubscript{POM} for their
small O:C and large H:C ratio (figure \ref{fig:p-el-ch-co1-res}).\\
To compare these patterns to those for HS, we present the same variables
in figure \ref{fig:p-el-ch-co1-res} B, but including various IHSS
reference HS and measured EAC and EDC values from
\citet{Aeschbacher.2012} for some of these. The H:C values of the HS are
separated by \textasciitilde0.5 from that of the peat samples, whereas
there is a rather large overlap in the O:C ratio between both sample
groups. One exception is the Pony Lake Fulvic Acid (FA) that has a H:C
ratio similar to our peat samples. Overall, the HS samples, with their
larger EAC and EDC, extent the gradient we observed for the peat
samples. We additionally labeled regions according to OM fractions
typically delineated in van-Krevelen diagrams \citep{Kim.2003}. The peat
samples spread between the cellulose and lignin regions, whereas the HS
samples are shifted more towards the lignin region due to their small
H:C ratios (except Pony Lake FA) (figure \ref{fig:p-el-ch-co1-res} b).

\begin{figure}[H]

{\centering \includegraphics[width=0.7\textwidth]{001-paper-main_files/figure-latex/p-el-ch-co1-res-1} 

}

\caption{Van-Krevelen-like plot for the peat samples analysed in this study. \textbf{a:} Points are scaled relative to the EAC and EDC, respectively, and coloured accoring to the sites the samples were taken from. \textbf{b:} The same as a, but including all samples from this study and various IHSS reference humic substances (red and small grey points) in addition to our peat samples (blue points). HS for which Aeschbacher et al. (2012) measured EAC and EDC data are filled red and scaled according to these values. HS for which this was not the case are represented as uniformly small grey points. Moreover, we highlighted regions commonly attributed to different OM fractions in van-Krevelen plots (Kim (2003)). PLFA is the Pony Lake Fulvic Acid reference HS. From this study, only samples with a potential contribution of iron to the EAC and EDC $\le\SI{100}{\micro\mol\per\gram\carbon}$ are shown and point sizes are scaled relative to the EAC\textsubscript{POM} and EDC\textsubscript{POM}, respectively.}\label{fig:p-el-ch-co1-res}
\end{figure}

This observed gradient is partly also supported by the regression
models. For the EAC\textsubscript{POM}, the H:C ratio has a negative
coefficient (\(\beta_\text{H:C}\in\){[}\(-2.58, -0.8\){]}; All reported
intervals are 95\%-posterior intervals for the models with all element
ratios and Gaussian coefficients, except if stated differently) and the
O:C ratio a positive coefficient
(\(\beta_\text{O:C}\in\){[}\(0.38, 2.27\){]}) for all models. For the
EDC\textsubscript{POM}, the O:C ratio only has a positive coefficient
(\(\beta_\text{O:C}\in\){[}\(0.7, 1.92\){]}) if the N:C and S:C ratio
are not included, whereas the coefficients' posterior intervals for the
H:C ratio broadly overlaps zero
(\(\beta_\text{H:C}\in\){[}\(-1.09, 0.37\){]}). If the N:C and S:C ratio
are included in the model, the coefficient for the O:C ratio
(\(\beta_\text{O:C}\in\){[}\(-0.64, 1.09\){]}) and H:C ratio
(\(\beta_\text{H:C}\in\){[}\(-0.54, 0.93\){]}) clearly overlap zero.\\
For the EAC\textsubscript{POM}, neither the N:C
(\(\beta_\text{N:C}\in\){[}\(-0.82, 0.86\){]}), nor the S:C ratio
(\(\beta_\text{S:C}\in\){[}\(-0.17, 0.74\){]}) have a clearly from zero
different coefficient. For the EDC\textsubscript{POM}, the S:C ratio is
also not clearly different from zero
(\(\beta_\text{S:C}\in\){[}\(-0.32, 0.7\){]}), whereas the model implies
a negative relation for the N:C ratio
(\(\beta_\text{N:C}\in\){[}\(-2.55, -0.63\){]}). Thus, it seems that the
EAC\textsubscript{POM} is linearly mainly related to the H:C and O:C
ratio, whereas the EDC\textsubscript{POM} is mainly related to the N:C
ratio.

\subsection{Relation to Molecular Structures}

MIR variables typically assigned to labile OM fractions tend to be
positively related to the EAC\textsubscript{POM} and
EDC\textsubscript{POM} and variables typically assigned to more
recalcitrant OM fractions tend to be negatively related to the
EAC\textsubscript{POM} and EDC\textsubscript{POM} (supporting figure
\ref{si-fig:p-mir-cor-res}). This is evident from positive correlations
with MIR variables representing cellulose C-O stretching, phenol C-O
stretching and O-H bending, carbonyl C=O stretching, cellulose and
phenol O-H stretching and negative correlations with MIR variables
representing aromatic C=C stretching, C-H bending, and lipid C-H
stretching
\citep{Stuart.2005, Cocozza.2003, Artz.2008, Kubo.2005, Schmidt.2006}.
However, the correlations are overall relatively small (maximum absolute
correlation \(=0.61\)).\\
The general patterns in the correlation spectra for the
EAC\textsubscript{POM} are similar to those for the
EDC\textsubscript{POM}. Some deviations are visible: For example the
EDC\textsubscript{POM} has a more negative relation to aromatic C=C
stretching, and a more positive relation to cellulose and phenol O-H
stretching, whereas the EAC\textsubscript{POM} is more strongly related
to carbonyl C=O stretching.\\
Variable selection during the regression analysis identified 6 and 2
variables as sufficient to predict the EAC\textsubscript{POM} and
EDC\textsubscript{POM}, respectively: The EAC\textsubscript{POM} is
positively related to carbonyl group C=O stretching and tends to be
negatively related to lipid C-H stretching, O-H stretching of unbonded
OH groups, and C-H bending of (potentially polysubstituted) aromatics
(table \ref{tab:t-mir-sel}). For lipid C-H stretching and unbonded O-H
stretching, two directly neighboring variables with partly contrasting
coefficients (positive and negative) were selected. However, their joint
relation to the EAC\textsubscript{POM} is clearly negative (supporting
figure \ref{si-fig:p-partial-dependence-res}). The
EDC\textsubscript{POM} tends to be positively related to O-H stretching
of intramoleculary bonded OH groups, probably of cellulose and phenols,
and tends to be negatively related to secondary amide N--H bending and
C--N stretching (table \ref{tab:t-mir-sel}).\\
Overall, this indicates that the EAC\textsubscript{POM} is positively
related to carbonyl groups and negatively related to structures more
abundant in decomposed peat (lipids, aromatics, unbonded OH groups), and
the EDC\textsubscript{POM} is positively related to structures more
abundant in undecomposed peat (intramolecular OH bonds in cellulose and
phenols).

\begin{table}

\caption{\label{tab:t-mir-sel}Assignment of MIR variables included in the projected regression models for the EAC\textsubscript{POM} and EDC\textsubscript{POM} using the filtered data set, respectively. "Wavenumber" represents the average bin position wavenumber value of the MIR variable that were selected. "Coefficient" are the estimated coefficients of the variables (mean and limits of the 95\% posterior intervals).}
\centering
\resizebox{\linewidth}{!}{
\begin{tabular}[t]{cll>{\raggedright\arraybackslash}p{4cm}>{\raggedright\arraybackslash}p{3cm}}
\toprule
\multicolumn{1}{c}{ } & \multicolumn{2}{c}{Coefficient} & \multicolumn{1}{c}{ } & \multicolumn{1}{c}{ } \\
\cmidrule(l{3pt}r{3pt}){2-3}
Wavenumber & EAC\textsubscript{POM} & EDC\textsubscript{POM} & Assigned structure & Reference\\
\midrule
830 & -0.13 (-0.28, 0.03) &  & C-H bending of di- or trisubstituted aromatics & \citet{Stuart.2005}\\
1530 &  & -0.23 (-0.68,0.17) & Secondary amide N–H bending and C–N stretching & \citet{Stuart.2005}\\
1720 & 0.41 (0.19, 0.62) &  & Carbonyl C=O stretching (carboxyls, esters, ketones - aliphatic and aromatic) & \citet{Cocozza.2003}, \citet{Stuart.2005}, \citet{Artz.2008}\\
2890 & -0.71 (-1.1, -0.28) &  & Lipid C-H stretching & \citet{Cocozza.2003}, \citet{Stuart.2005}, \citet{Artz.2008}\\
2910 & 0.1 (-0.27, 0.45) &  & Lipid C-H stretching & \citet{Cocozza.2003}, \citet{Stuart.2005}, \citet{Artz.2008}\\
\addlinespace
3370 &  & 0.34 (-0.09,0.75) & O-H  stretching of bonded OH groups (cellulose, phenols) & \citet{Stuart.2005} \citet{Kubo.2005} \citet{Schmidt.2006}\\
3660 & 1.45 (-0.84, 3.58) &  & O-H stretching of unbonded OH groups & \citet{Stuart.2005}, \citet{Kubo.2005}, \citet{Schmidt.2006}\\
3670 & -1.88 (-4.02, 0.42) &  & O-H stretching of unbonded OH groups & \citet{Stuart.2005}, \citet{Kubo.2005}, \citet{Schmidt.2006}\\
\bottomrule
\end{tabular}}
\end{table}

\subsection{Predictive Accuracy of the MIRS-Based Models in Comparison
to Regression Models Based on Element Ratios}

The predictive performances of the different modeling approaches
(MIRS-based vs element ratio-based) do not differ considerably and are
several times larger than the standard deviation of the distributions
for the respective replicate measurements. The MIRS-based models had a
median 10-fold CV-RMSE \(23.5\) and
\(\SI{4.4}{\micro\mol\per\g\carbon}\) smaller and larger than the
element ratio-based models for the EAC\textsubscript{POM} and
EDC\textsubscript{POM}, respectively. The models with the least median
10-fold CV-RMSE for each variable had a 10-fold CV-RMSE of 250.4
{[}101.3, 509{]} and 160.5 {[}58.1, 279.7{]}
\(\si{\micro\mol\per\g\carbon}\) for the EAC\textsubscript{POM} and
EDC\textsubscript{POM}, respectively. In comparison to this, the median
standard deviation from the estimated distribution for the measured
values is 36.2 and \(\SI{21.2}{\micro\mol\per\g\carbon}\). In spite of
the similar predictive performance of both models for the
EAC\textsubscript{POM}, the MIRS-based model is clearly less biased than
the element ratio-based model, whereas both models for the
EDC\textsubscript{POM} are clearly biased (supporting figure
\ref{si-fig:p-y-yhat-res}).\\
Using derivative spectra, data subsets, or PLSR instead of Bayesian
regularization neither yield considerably worse, nor better models in
terms of their training predictive accuracy (supporting figure
\ref{si-fig:p-cal-elpd-res}).

\section{Discussion}

To answer how peat chemistry relates to its EAC\textsubscript{POM} and
EDC\textsubscript{POM} and how decomposition changes both, we use our
results to develop a conceptual model describing how vegetation
chemistry and intensity of aerobic decomposition control peat
electrochemical properties via the amount of polymeric phenols and
quinones.\\
The analyzed peat samples cover a globally representative range of mid
to high latitude peat properties and degrees of decomposition (table
\ref{tab:t-ranges-sites}). The element ratio data are within the ranges
reported by several larger compilations of peat chemical properties
\citep{Moore.2018, Leifeld.2020, Wang.2015b, Loisel.2014, Tipping.2016}.
The same is true for C\(_\textrm{OX}\) values
\citep{Worrall.2016b, Moore.2018, Leifeld.2020}. Overall, we are
confident that our findings and interpretations should hold for a broad
range of peat materials and thus are generalizable.

\subsection{\texorpdfstring{Quinones and Phenols are Main Contributors
to Peat EAC\textsubscript{POM} and
EDC\textsubscript{POM}}{Quinones and Phenols are Main Contributors to Peat EAC and EDC}}

Our results point towards quinones and phenols as main contributors to
the EAC\textsubscript{POM} and EDC\textsubscript{POM} of peat,
respectively. EAC\textsubscript{POM} and EDC\textsubscript{POM} values
are large along a H:C-O:C gradient with large O:C and large H:C ratio at
the one end and small O:C and small H:C ratio at the other end (figure
\ref{fig:p-el-ch-co1-res}). Based on commonly delineated H:C-O:C regions
in van-Krevelen diagrams \citep{Kim.2003}, we assume that this gradient
characterizes material rich in polymeric quinones and phenols. This
interpretation is also supported by the positive relation to carbonyl
groups which are characteristic of polymeric quinones and phenols
\citep{ElMansouri.2007} (supporting figure \ref{si-fig:p-mir-cor-res}).
In addition, the more pronounced negative correlation of the
EDC\textsubscript{POM} to the N:C ratio in comparison to the
EAC\textsubscript{POM} can be explained with the larger susceptibility
of phenols towards decomposition in contrast to quinones which are
formed by partial oxidation of phenols
\citep{Fenner.2011, Aeschbacher.2012, Bolton.2018}. We expected this
finding since quinones and phenols have been identified as major
contributors to OM EAC and EDC in general
\citep{Ratasuk.2007, Aeschbacher.2010, Aeschbacher.2012}.\\
We did not find clear relations of the N:C or S:C ratio to the
EAC\textsubscript{POM} and EDC\textsubscript{POM} that would point
towards a large contribution of non-quinone moieties to the
EAC\textsubscript{POM} or EDC\textsubscript{POM}. In fact, the N:C ratio
is negatively related to the EDC. A small non-quinone EAC and EDC is in
line with the relative small S and N contents of the peat samples and
the neutral pH value our measurements were standardized to
\citep{Fimmen.2007, HernandezMontoya.2012}; non-quinone
EAC\textsubscript{POM} and EDC\textsubscript{POM} would likely be larger
with smaller pH values
\citep{Fimmen.2007, HernandezMontoya.2012, Aeschbacher.2012}. Moreover,
it is likely that any relation between the N:C and S:C ratio and
non-quinone EAC\textsubscript{POM} and EDC\textsubscript{POM} is
confounded by the dependence of these element ratios and the quinone
EAC\textsubscript{POM} and EDC\textsubscript{POM} on the degree of
decomposition \citep{Biester.2014} (see below).

\subsection{\texorpdfstring{Vegetation Chemistry and Decomposition Cause
the Decoupling of Peat EAC\textsubscript{POM} and
EDC\textsubscript{POM}}{Vegetation Chemistry and Decomposition Cause the Decoupling of Peat EAC and EDC}}

The decoupling of peat EAC\textsubscript{POM} and EDC\textsubscript{POM}
(figure \ref{fig:p-el-comparison-res} b) can be explained by the joint
effects of vegetation polymeric phenol contents and transformation of
phenols to quinones during decomposition \citep{Aeschbacher.2012}. It
has been suggested that undecomposed plant-derived polymeric aromatics,
such as lignin, have large contents of phenols, but small contents of
quinones, which results in an initially large EDC, but small EAC
\citep{Aeschbacher.2012}. Decomposition and oxidation of the polymeric
phenols decrease their fraction, but can increase the relative (and
absolute) amount of quinones
\citep{Aeschbacher.2012, Bolton.2018, LaCroix.2020}. These mechanisms
have four important implications: First, undecomposed samples have a
maximum EDC, second, this maximum EDC varies depending on the vegetation
polymeric phenol content, third, decomposition of polymeric phenols
decreases the EDC and increases the EAC \citep{Aeschbacher.2012}, and
fourth, the initial EDC defines the maximum potential EAC of a sample.
This explains why a decoupling of the EAC and EDC can be observed for HS
and DOM \citep{Aeschbacher.2012}. Our measurements fit into the EDC-EAC
gradient observed for HS and DOM (supporting figure
\ref{si-fig:p-el-eac-edc2-res}) and we therefore conclude that the
decoupling of peat EAC\textsubscript{POM} and EDC\textsubscript{POM} is
caused by the same mechanisms.

\subsection{The Degree of Decomposition Confounds How the H:C ratio
Represents Phenols and Quinones}

In contrast to \citet{Aeschbacher.2012}, we found only a weak relation
between the EAC\textsubscript{POM} and the H:C ratio. In
\citet{Aeschbacher.2012}, a large H:C ratio indicates a smaller
polymeric phenol and quinone content, but also a larger polysaccharide
content because the H:C ratio is related to the O:C ratio (figure
\ref{fig:p-el-ch-co1-res}). In contrast, for our peat POM samples, a
large H:C ratio is not indicative for a larger polysaccharide content
because it could also represent strongly decomposed samples with high
lipid content (figure \ref{fig:p-el-ch-co1-res} a). This confounds how
the H:C ratio represents polymeric phenols and quinones because peat
samples with the same large H:C ratio tend to have a smaller
EAC\textsubscript{POM} if they contain a larger amount of lipids and
smaller amount of polysaccharides (smaller O:C ratio). This indicates
that strongly decomposed, lipid rich, peat has fewer polymeric phenols
and quinones than undecomposed peat. In other terms, for the peat POM
samples, the same H:C ratio can represent samples with either large O:C
ratio or small O:C ratio, corresponding either to relatively
undecomposed samples rich in polysaccharides and phenols and quinones or
to strongly decomposed samples rich in lipids, but depleted in phenols
and quinones \citep{Kim.2003, Leifeld.2012, Bader.2018}. This
interpretation is also supported by the joint relation of the H:C and
O:C ratio to the EAC\textsubscript{POM} in our regression models and by
negative coefficients for MIRS variables representing lipids (table
\ref{tab:t-mir-sel}, supporting figure \ref{si-fig:p-mir-cor-res}).
Thus, since differences in the O:C ratio are linked to differences in
the EAC\textsubscript{POM}, and since such samples are present within
our data set, we found a weaker relation of the EAC\textsubscript{POM}
to the H:C ratio. It is likely that the same pattern caused the relative
weak relation \citet{Tan.2017} observed for the EAC of HS to the H:C
ratio.

\subsection{Decomposition indicators are poorly related to phenols and
quinones}

We argue that many peat decomposition indicators are only weakly related
to the EAC\textsubscript{POM} and EDC\textsubscript{POM} because they
are no optimal descriptors of the polymeric phenol and quinone content.
The O:C ratio is an indicator of the polysaccharide content and strongly
related to all other decomposition indicators we analyzed (Pearson
correlations of the O:C ratio with the N:C ratio,
HI\textsubscript{1630/1090}, and C\textsubscript{OX} are -0.68, -0.82,
and 0.87, respectively). If we focus on how these indicators describe
the degree of decomposition, there is the same confounding as described
above, but viewed from a different perspective: for strongly decomposed
peat (small O:C ratio), the indicators cannot distinguish between peat
with large content of polymeric quinones and phenols on the one side
(small H:C ratio) and strongly decomposed peat rich in lipids, but
depleted in phenols and quinones (large H:C ratio). For this reason, the
EAC\textsubscript{POM} can vary considerably across a broad range of
HI\(_\text{1630/1090}\) and N:C ratios, as observed in our peat samples
(supporting figure \ref{si-fig:p-el-variables1-res}).\\
For the EDC\textsubscript{POM}, this confounding effect is less relevant
since the EDC\textsubscript{POM} is negatively affected by
decomposition, irrespective if the resulting peat is rich in lipids or
polymeric quinones, as mentioned above \citep{Fenner.2011, Bolton.2018}.
This is evident from the stronger relation of the EDC\textsubscript{POM}
to the N:C ratio (\(\rho=-0.69\)). Nevertheless, since samples with a
small O:C ratio that are rich in polymeric quinones and phenols have a
larger EDC\textsubscript{POM}, and since the HI\(_\text{1630/1090}\) and
C\(_\text{OX}\) do separate these samples less clearly from the lipid
rich samples than the N:C ratio (supporting figure
\ref{si-fig:p-el-variables1-res}), the relation of both the
HI\(_\text{1630/1090}\) and C\(_\text{OX}\) is also weak for the
EDC\textsubscript{POM}.

\subsection{Which Factors Determine the Polymeric Phenol and Quinone
Content of Decomposed POM?}

Which factors determine if a sample with a larger degree of
decomposition is either rich in polymeric quinones and phenols --- thus
having a larger EAC\textsubscript{POM} and EDC\textsubscript{POM} --- or
rich in lipids --- thus having the minimal EAC\textsubscript{POM} and
EDC\textsubscript{POM}? Both, differences in vegetation chemistry and
decomposition, may result in the observed differences in lipid versus
polymeric quinone and phenol content.\\
Even though we cannot definitely disentangle both factors based on our
data --- we neither have complete information about the peat forming
vegetation, nor the actual predominant decomposition processes --- there
is some evidence how both contribute. First of all, the samples with the
smallest H:C ratio probably contain larger amounts of wood and root
remains from trees and shrubs (e.g.~both Lutose sites, especially the
deepest samples \citep{Heffernan.2020}; P. Brunswick
\citep{Broder.2012}; Mer Bleue, deepest sample \citep{Elliott.2012})
which is a plausible explanation for their large EAC\textsubscript{POM}.
In contrast, samples with approximately the same O:C ratio, but a larger
H:C ratio are probably formed by sedges and (minerotrophic)
\emph{Sphagnum} mosses (samples from both both fen sites (Touxi,
Dongtu), or from the two Patagonian bogs SKY I-1 and SKY I-6) and likely
strongly decomposed (Touxi, Dongtu) or known to be strongly decomposed
(Both SKY I sites and SKY II, \citet{Broder.2012},
\citet{Mathijssen.2019}). Second, intense aerobic decomposition of peat
under drainage results in a larger H:C ratio, whereas less oxic
conditions result in a decrease of the H:C ratio during decomposition
\citep{Leifeld.2012}. A plausible and likely explanation therefore is
that aerobic decomposition of POM initially rich in polymeric phenols
results in a large amount of quinones and therefore a large
EAC\textsubscript{POM}, whereas \emph{intense} aerobic decomposition of
POM that already had small initial amounts of polymeric phenols results
in a large relative amount of lipids. Conversely, anaerobic
decomposition may conserve initial polymeric phenols and quinones and
thus EAC\textsubscript{POM} and EDC\textsubscript{POM} (see below). This
means that both vegetation chemistry and intensity of aerobic
decomposition contribute to the observed pattern.

\subsection{\texorpdfstring{Differential Effects of Decomposition on
Peat
EAC\textsubscript{POM}}{Differential Effects of Decomposition on Peat EAC}}

The hypothesis that vegetation chemistry and decomposition intensity
together change peat phenols and quinone content can help resolving the
apparent contradiction we have produced: Partial oxidation of phenols to
quinones during decomposition increases the EAC\textsubscript{POM}
\citep{Aeschbacher.2012, Walpen.2018, Tan.2017}, but typical
MIRS-derived decomposition indicators such as the amount of lipids,
unbonded OH groups, and aromatic backbone structures
\citep{Cocozza.2003, Artz.2008} are negatively related to the
EAC\textsubscript{POM} (table \ref{tab:t-mir-sel}). According to our
conceptual understanding, the contradiction is only apparent: Aerobic
decomposition of POM initially rich in polymeric phenols and quinones
increases the amount of quinones \citep{Aeschbacher.2012}, as indicated
by larger amounts of carbonyl groups (table \ref{tab:t-mir-sel}).
Intense aerobic decomposition of POM with initially low amounts of
polymeric phenols and quinones (and hence larger amounts of
polysaccharides) results in the mineralization of polysaccharides,
phenols and quinones and the accumulation of lipids
\citep{Fenner.2011, Leifeld.2012}. Thus, if a specific peat contains
large amounts of lipids and has an amorphous structure, this indicates
that it also contains low amounts of polymeric phenols and quinones
because it experienced intense decomposition.

\subsection{\texorpdfstring{A conceptual model for peat
EAC\textsubscript{POM} and
EDC\textsubscript{POM}}{A conceptual model for peat EAC and EDC}}

To summarize our findings, we propose the conceptual model shown in
figure \ref{fig:conceptual1}. We hypothesize that both the polymeric
phenol content of the peat forming vegetation and the intensity of
decomposition processes are the most important factors controlling the
EDC\textsubscript{POM} and EAC\textsubscript{POM} of peat. Moreover,
both factors likely interact since peat with initially large amounts of
polymeric phenols has a smaller decomposition rate
\citep{Bengtsson.2018}.\\
Wood and roots from trees and shrubs are probably the plant remains with
the largest fraction of polymeric phenols
\citep{Benner.1984, Williams.1998, Strakova.2010, Hodgkins.2018},
whereas other vascular plants and mosses have variable and partly
smaller amounts of polymeric phenols
\citep{Williams.1998, Scheffer.2001, Strakova.2010, Bengtsson.2018, Zak.2019}.
For this reason, peat with larger contributions of wood or roots
e.g.~from shrubs (and potentially some graminoid or moss species) likely
has the largest initial EDC\textsubscript{POM} and upon decomposition
the largest potential EAC\textsubscript{POM} (figure
\ref{fig:conceptual1} and supporting figure \ref{si-fig:conceptual2}).
We propose that standardized measurements of electrochemical properties
for different peat forming species are required to provide a
quantitative basis for this hypothesis.\\
Under anoxic conditions, low phenol oxidase activities and the effects
this has on other enzymes required for biomass breakdown
\citep{Fenner.2011} make peat material keep its initial
EDC\textsubscript{POM} and EAC\textsubscript{POM}. In addition, quinone
formation by partial oxidation is limited under such conditions, such
that the EDC\textsubscript{POM} should be relative large, whereas the
EAC\textsubscript{POM} remains smaller (figure \ref{fig:conceptual1} and
supporting figure \ref{si-fig:conceptual2}). A factor that may decrease
both the EAC\textsubscript{POM} and EDC\textsubscript{POM} under such
conditions are condensation reactions
\citep{Hotta.2002, Uchimiya.2009, Bolton.2018, Zhao.2020, Olk.2006, Heitmann.2006, Yu.2016}.
Conversely, faster degradation of polysaccharides may increase the
EAC\textsubscript{POM} and EDC\textsubscript{POM} since this results in
a relative increase of phenols and quinones \citep{Benner.1984}.
However, it remains currently unclear to which extent both factors may
play a role under anoxic conditions. Thus, we assume that under anoxic
conditions the initial vegetation properties largely control peat
electrochemical properties.\\
Under oxic conditions, increased phenol oxidase activities result in an
oxidative transformation of polymeric phenols to quinones
\citep{Fenner.2011, Schellekens.2015, Bolton.2018}. This can increase
the EAC\textsubscript{POM} and decrease the EDC\textsubscript{POM}. If
under these oxic conditions e.g.~large temperatures, large
concentrations of nutrients, large amounts of labile OM, high pH values,
or other factors further facilitate the mineralization of polymeric
phenols \citep{Bragazza.2007, Fenner.2011, Kang.2018c, Bowring.2020},
coupling reactions
\citep{Hotta.2002, Johnson.2015, Bolton.2018, Zhao.2020}, and other
condensation reactions
\citep{Bolton.2018, Olk.2006, Heitmann.2006, Yu.2016}, both the
EAC\textsubscript{POM} and EDC\textsubscript{POM} may decrease (figure
\ref{fig:conceptual1} and supporting figure \ref{si-fig:conceptual2}).
Intense break-down of cell wall structures (as implied by our regression
models for the EDC\textsubscript{POM} and EAC\textsubscript{POM}) may
has an important role since it increases the surface of the polymeric
phenols exposed for oxidation or condensation reactions
\citep{Tsuneda.2001}. Just as under anoxic conditions, a decrease in the
EAC\textsubscript{POM} and EDC\textsubscript{POM} may be offset by
faster mineralization of polysaccharides than polymeric phenols
\citep{Benner.1984}. However, again it remains currently unclear to
which extent this factor and also condensation reactions play a role.
Thus, we assume that oxic conditions increase the EAC\textsubscript{POM}
at the expense of the EDC\textsubscript{POM}. The initial content of
polymeric phenols defines the maximum EAC\textsubscript{POM} peat can
attain throughout this process which results in variable responses of
the peat EAC\textsubscript{POM} and EDC\textsubscript{POM} to
decomposition. Intense aerobic decomposition, especially of material
with low initial amounts of polymeric phenols, may even decrease the EAC
due to mineralization and condensation of polymeric phenols.

\begin{figure}[H]

{\centering \includegraphics[width=1\linewidth]{./../figures/conceptual_vegetation_decomposition1} 

}

\caption{Conceptual description of the assumed effects of vegetation chemistry (polymeric phenol content) and decomposition pathways and intensity on peat chemistry and its EAC and EDC. Note that the assignment of plant taxa and their polymeric phenol contents is only an example as the chemical properties may be highly diverse within taxa \citet{Bengtsson.2018}.}\label{fig:conceptual1}
\end{figure}

\subsection{\texorpdfstring{MIRS-Based Regression Models can Predict
Peat
EAC\textsubscript{POM}}{MIRS-Based Regression Models can Predict Peat EAC}}

We suggest that our modeling approach is a proof-of-concept that peat
EAC\textsubscript{POM} can be predicted from MIRS. The MIRS-based model
had the smallest average RMSE, albeit the RMSE for both the MIRS-based
model and the element ratio-based model were relatively large and their
95\%-posterior intervals broad and strongly overlapping. During
graphical model validation, we observed that the element ratio-based
model had a considerably larger bias than the MIRS-based model
(supporting figure \ref{si-fig:p-y-yhat-res}), suggesting that the
MIRS-based model overall is more robust.\\
In contrast to this, both element ratio-based and MIRS-based regression
models failed to adequately capture the variability in the
EDC\textsubscript{POM}. Since the EDC\textsubscript{POM} of our samples
in their oxidized state was smaller, the relative large predictive
uncertainties turn both models unsuitable for practical applications.

\section{Conclusions}

Our research question was how peat chemistry relates to its
EAC\textsubscript{POM} and EDC\textsubscript{POM} and how decomposition
changes both. Based on our results, we hypothesize a conceptual model
that describes how vegetation chemistry and intensity of aerobic
decomposition control peat EAC\textsubscript{POM} and
EDC\textsubscript{POM}. Undecomposed peat formed by vegetation rich in
polymeric phenols has the largest EDC\textsubscript{POM}. Decomposition
of such material results in peat with the largest
EAC\textsubscript{POM}, but decreases the EDC\textsubscript{POM}. In
contrast, peat formed by vegetation with small amounts of polymeric
phenols generally has a smaller EAC\textsubscript{POM} and
EDC\textsubscript{POM}. Especially for such material, intense
decomposition not only decreases the EDC\textsubscript{POM}, but
potentially also the EAC\textsubscript{POM}. This model can plausibly
explain the large variability in the relation of the
EAC\textsubscript{POM} and EDC\textsubscript{POM} to peat chemical
properties, decomposition indicators, and molecular structures, as well
as the high intra-site variability and decoupling of the
EAC\textsubscript{POM} and EDC\textsubscript{POM}. Finally, we provide a
proof-of-concept that MIRS-based regression models may be at least
suitable as screening tools to predict peat EAC\textsubscript{POM}.\\
Our study has four inherent limitations: First, it is difficult to
completely exclude contributions of iron to our estimates for the
EAC\textsubscript{POM} and EDC\textsubscript{POM}.Second, we derived our
conceptual understanding on purely observational data without
experimental control. Third, we had only limited data on
palaeovegetation and thus could only partly establish direct links
between vegetation and peat chemistry.\\
Nevertheless, our results imply that peat EAC\textsubscript{POM} and
EDC\textsubscript{POM} can be spatially and temporally highly variable
and that it is difficult to predict based on peat bulk properties. This
furthermore implies that the potential for CH\(_4\) suppression due to
POM reduction may be similarly variable and difficult to predict.\\
Therefore, spatially resolved measurements or the incorporation of our
hypothesized conceptual understanding into process models are required
for the accurate quantification of peat EAC\textsubscript{POM} and
EDC\textsubscript{POM} and their potential effects on redox processes,
particularly CH\(_4\) formation.

\section*{Open Research}
\addcontentsline{toc}{section}{Open Research}

Data and code to reproduce this document are available as research
compendium on Zenodo and GitHub {[}--- todo: add link to GitHub repo,
add Zenodo DOI, add citation, add license{]}. In addition, the
MIRS-based reference models based on Bayesian regularization for both
the EAC\textsubscript{POM} and EDC\textsubscript{POM} are available via
the R package irpeat \citep{Teickner.2020b}. These models can be used
for predictions with own MIRS data. {[}For review purposes, the
reproducible research compendium including all necessary data and code
to reproduce the findings presented in the manuscript can be found at
the following link: https://uni-muenster.sciebo.de/s/SbbFgA8pOI4JKW9{]}

\section*{Author contributions}
\addcontentsline{toc}{section}{Author contributions}

HT (conceptualization, data curation, formal analysis, investigation,
methodology, software, validation, visualization, writing - original
draft, writing -- review \& editing). CG (conceptualization,
investigation, methodology, writing -- review \& editing), KHK
(conceptualization, funding acquisition, methodology, project
administration, resources, supervision, writing - original draft,
writing -- review \& editing).

\section*{Acknowledgements}
\addcontentsline{toc}{section}{Acknowledgements}

For their support during sample collection/provision, we would like to
thank Svenja Agethen (DE), Werner Borken (SKY I-1, SKY I-6), Tanja
Broder (PBR, SKY II, LT, MK), Mariusz Gałka (LT, MK), Liam Heffernan
(LP, LB), Norbert Hölzel (KR, ISH), Annkathrain Hömberg (TX, DT), Tim
Moore (MB), Sindy Wagner (BB), Tim-Martin Wertebach (KR, ISH), and
Zhi-Guo Yu (TX, DT).\\
Analyses of this study were carried out in the laboratory of the
Institute of Landscape Ecology. Svenja Agethen and Michael Sander
provided analytical support. The assistance of Ulrike Berning-Mader,
Madeleine Supper, Victoria Ratachin, and numerous student assistants is
greatly acknowledged. We thank Dr.~Hendrik Wetzel, Fraunhofer Institute
for Applied Polymer Research, Dept. Starch Modification/Molecular
Properties, Potsdam, Germany, for analysis of O and H. The workflow was
reproduced by the Reproducible Research Support Service of the
University of Münster.\\
This Study was funded by the Deutsche Forschungsgemeinschaft (DFG,
German Research Foundation) grant no. KN 929/12-1 to Klaus-Holger Knorr;
Chuanyu Gao was supported by the Youth Innovation Promotion Association
CAS (No.~2020235).

\nocite{Frolking.2011, Limpens.2008, Myhre.2013, Moore.1989, Yavitt.1997, Chaudhary.2020, Anisimov.2007, Koven.2011, Blodau.2011, Klupfel.2014, Gao.2019, Lau.2016, Walpen.2018b, Blodau.2007, Ratasuk.2007, Aeschbacher.2010, Aeschbacher.2012, Fimmen.2007, HernandezMontoya.2012, Tan.2017, Walpen.2018, LaCroix.2020, Uchimiya.2009, Keller.2013, Lau.2015, Worrall.2017, Kim.2003, Leifeld.2012, Bader.2018, Moore.2018, Leifeld.2020, Cocozza.2003, Artz.2008, Hodgkins.2018, Tfaily.2014, Lv.2018, Biester.2014, Broder.2012, Drollinger.2020, Yuan.2018, South.2017, Rydin.2013, Danielson.2011, Fick.2017, Sagerfors.2008, Wertebach.2016, Larina.2013, Heffernan.2020, Elliott.2012, Mathijssen.2019, Aeschbacher.2011, Tamura.1974, Masiello.2008, Worrall.2016b, Teickner.2020, Teickner.2020b, signaldevelopers.2014, Stuart.2005, Engel.2013, Poisot.2011, HuffmanLaboratories.NA, Ferragina.2015, Piironen.2020, Roberts.2017, Bellocchi.2010, Kubo.2005, Schmidt.2006, Wang.2015b, Loisel.2014, Tipping.2016, ElMansouri.2007, Fenner.2011, Bolton.2018, Bengtsson.2018, Benner.1984, Williams.1998, Strakova.2010, Scheffer.2001, Zak.2019, Hotta.2002, Zhao.2020, Olk.2006, Heitmann.2006, Yu.2016, Schellekens.2015, Bragazza.2007, Kang.2018c, Bowring.2020, Johnson.2015, Tsuneda.2001}

\bibliography{references}

\nocitesi{Beleites.2020, Carpenter.2017, StanDevelopmentTeam.2020, Gabry.2019b, Gelman.2014b, Gabry.2019, Mevik.2019, Goodrich.2020, Piironen.2017c, Piironen.2019, RCoreTeam.2020, Wickham.2016, Wilke.2019, Slowikowski.2020, Pedersen.2019, Pebesma.2018b, Pebesma.2005, Bivand.2020, Hijmans.2020, Wickham.2020e, Henry.2020, Bache.2014, Wickham.2020f, Pebesma.2016, Ucar.2019, Teickner.2020e, Moore.2007, Scheffer.2000, Kruschke.2015, Baath.2018}
\bibliographystylesi{apacite}
\bibliographysi{references}



\end{document}
