%% March 2018
%%%%%%%%%%%%%%%%%%%%%%%%%%%%%%%%%%%%%%%%%%%%%%%%%%%%%%%%%%%%%%%%%%%%%%%%%%%%
% AGUJournalTemplate.tex: this template file is for articles formatted with LaTeX
%
% This file includes commands and instructions
% given in the order necessary to produce a final output that will
% satisfy AGU requirements, including customized APA reference formatting.
%
% You may copy this file and give it your
% article name, and enter your text.
%
%
% Step 1: Set the \documentclass
%
% There are two options for article format:
%
% PLEASE USE THE DRAFT OPTION TO SUBMIT YOUR PAPERS.
% The draft option produces double spaced output.
%

%% To submit your paper:
\documentclass[draft,linenumbers]{agujournal2018}
\usepackage{apacite}
\usepackage{url} %this package should fix any errors with URLs in refs.
%%%%%%%
% As of 2018 we recommend use of the TrackChanges package to mark revisions.
% The trackchanges package adds five new LaTeX commands:
%
%  \note[editor]{The note}
%  \annote[editor]{Text to annotate}{The note}
%  \add[editor]{Text to add}
%  \remove[editor]{Text to remove}
%  \change[editor]{Text to remove}{Text to add}
%
% complete documentation is here: http://trackchanges.sourceforge.net/
%%%%%%%


%% Enter journal name below.
%% Choose from this list of Journals:
%
% JGR: Atmospheres
% JGR: Biogeosciences
% JGR: Earth Surface
% JGR: Oceans
% JGR: Planets
% JGR: Solid Earth
% JGR: Space Physics
% Global Biogeochemical Cycles
% Geophysical Research Letters
% Paleoceanography and Paleoclimatology
% Radio Science
% Reviews of Geophysics
% Tectonics
% Space Weather
% Water Resources Research
% Geochemistry, Geophysics, Geosystems
% Journal of Advances in Modeling Earth Systems (JAMES)
% Earth's Future
% Earth and Space Science
% Geohealth
%
% ie, \journalname{Water Resources Research}

\journalname{Global Biogeochemical Cycles}


\usepackage{siunitx}
\DeclareSIQualifier\carbon{C}
\DeclareSIQualifier\sample{sample}
\sisetup{
	qualifier-mode = subscript
	}
\DeclareSIUnit\wn{\cm\tothe{-1}}
\usepackage{hyperref}
\usepackage{soulutf8}
\usepackage{booktabs}
\usepackage{longtable}
\usepackage{array}
\usepackage{multirow}
\usepackage{wrapfig}
\usepackage{float}
\usepackage{colortbl}
\usepackage{pdflscape}
\usepackage{tabu}
\usepackage{threeparttable}
\usepackage{threeparttablex}
\usepackage[normalem]{ulem}
\usepackage{makecell}
\usepackage{xcolor}
\pagestyle{plain}
\usepackage{xr}
\externaldocument[main-]{001-paper-main}

\begin{document}

%% ------------------------------------------------------------------------ %%
%  Title
%
% (A title should be specific, informative, and brief. Use
% abbreviations only if they are defined in the abstract. Titles that
% start with general keywords then specific terms are optimized in
% searches)
%
%% ------------------------------------------------------------------------ %%

% Example: \title{This is a test title}

\title{Supporting Information for: ``Electrochemical Properties of Peat
Particulate Organic Matter on a Global Scale: Relation to Peat Chemistry
and Degree of Decomposition''}

%% ------------------------------------------------------------------------ %%
%
%  AUTHORS AND AFFILIATIONS
%
%% ------------------------------------------------------------------------ %%

% Authors are individuals who have significantly contributed to the
% research and preparation of the article. Group authors are allowed, if
% each author in the group is separately identified in an appendix.)

% List authors by first name or initial followed by last name and
% separated by commas. Use \affil{} to number affiliations, and
% \thanks{} for author notes.
% Additional author notes should be indicated with \thanks{} (for
% example, for current addresses).

% Example: \authors{A. B. Author\affil{1}\thanks{Current address, Antartica}, B. C. Author\affil{2,3}, and D. E.
% Author\affil{3,4}\thanks{Also funded by Monsanto.}}

\authors{
Henning Teickner
\affil{1}
Chuanyu Gao
\affil{1, 2}
Klaus-Holger Knorr
\affil{1}
}


% \affiliation{1}{First Affiliation}
% \affiliation{2}{Second Affiliation}
% \affiliation{3}{Third Affiliation}
% \affiliation{4}{Fourth Affiliation}

\affiliation{1}{ILÖK, Ecohydrology and Biogeochemistry Group, University of Münster,
Heisenbergstr. 2, 48149 Münster, Germany}
\affiliation{2}{Key Laboratory of Wetland Ecology and Environment, Northeast Institute
of Geography and Agroecology, Chinese Academy of Sciences, Shengbei
Street 4888, 130102, Changchun, China}
%(repeat as many times as is necessary)

%% Corresponding Author:
% Corresponding author mailing address and e-mail address:

% (include name and email addresses of the corresponding author.  More
% than one corresponding author is allowed in this LaTeX file and for
% publication; but only one corresponding author is allowed in our
% editorial system.)

% Example: \correspondingauthor{First and Last Name}{email@address.edu}
\correspondingauthor{Klaus-Holger Knorr}{klaus-holger.knorr@uni-muenster.de}
\correspondingauthor{Chuanyu Gao (Co-corresponding author)}{gaochuanyu@iga.ac.cn}

%% Keypoints, final entry on title page.

%  List up to three key points (at least one is required)
%  Key Points summarize the main points and conclusions of the article
%  Each must be 100 characters or less with no special characters or punctuation

% Example:
% \begin{keypoints}
% \item	List up to three key points (at least one is required)
% \item	Key Points summarize the main points and conclusions of the article
% \item	Each must be 100 characters or less with no special characters or punctuation
% \end{keypoints}

\begin{keypoints}
\item 
\end{keypoints}

%% ------------------------------------------------------------------------ %%
%
%  ABSTRACT
%
% A good abstract will begin with a short description of the problem
% being addressed, briefly describe the new data or analyses, then
% briefly states the main conclusion(s) and how they are supported and
% uncertainties.
%% ------------------------------------------------------------------------ %%

%% \begin{abstract} starts the second page

\begin{abstract}

\end{abstract}
\renewcommand{\thefigure}{S\arabic{figure}} 
\renewcommand{\thetable}{S\arabic{table}}

\pagestyle{plain}

\section{Graphical Abstract}

\begin{figure}[H]

{\centering \includegraphics[width=0.6\textwidth]{./../figures/graphical_abstract} 

}

\caption{Graphical abstract: Description: This study addresses the relation between peat chemical properties (element contents, mid infrared spectra) and electrochemical properties and the role of decomposition for both based on a global data set. Peat electrochemical properties (electron accepting capacity - EAC, electron donating capacity - EDC) control peatland methane formation. Phenols are main contributors to the EDC and quinones to the EAC. Vegetation chemistry and intensity of decomposition are major controls of peat EAC and EDC. Peat with initial large amount of phenols initially have a large EDC and upon decomposition a larger EAC since phenols are transformed to quinones. Samples with initially smaller amount of phenols cannot attain the same EAC/EDC levels, even though decomposition leads to similar changes. This conceptual model can explain the large within and between-site variability of the EAC and EDC we observed, as well as the complex relations to indicators of peat chemistry and decomposition indicators.}\label{fig:graphical-abstract}
\end{figure}

\section{Supplementary Methods}

\subsection{Mid Infrared Spectroscopy}

All spectra were linearly interpolated to a resolution of 1 cm\(^{-1}\).
To remove artifacts caused by CO\(_2\), all spectra were linearly
interpolated in the region 2290 to 2400 cm\(^{-1}\). Subsequently,
baseline correction was performed using a rubberband algorithm (based on
a convex hull procedure combined with smoothing splines)
\citep{Beleites.2020}. After this, spectra were clipped by 20 and 10
wavenumber units at the start and end, to the range 810 to
\SI{3990}{\wn}, and baseline correction was performed a second time to
fully align the start and end points of the spectra. Finally, all
spectra were normalized to unit intensity sum.

\subsection{Element Ratio-Based Regression Models}

In addition to the models described in the main text, we evaluated if
using robust regression (coefficients are assumed to follow a Student-t
distribution instead of a Gaussian distribution) has an effect on the
model fit since several samples did not fit in the overall pattern.\\
We applied several transformations to facilitate model fitting and
defining prior distributions: The EAC\textsubscript{POM} and
EDC\textsubscript{POM} values were divided by their maximal values
(i.e.~after transformation all values were \(\le1\)) and a small
constant of 0.05 was added to the scaled EDC\textsubscript{POM} values
to facilitate convergence. All predictor variables were centered (such
that their mean value is 0 and the model's intercept represents average
samples) and scaled (such that their standard deviation is 1).\\
We used a Normal prior distribution with a mean of log(0.5) and a
standard deviation of 0.1 amounting to a 95\%-confidence interval of the
prior intercept on the link function scale of approximately 0.34 to 0.76
which loosely matched our expectations of the position of the samples'
mean following our normalization procedure. We also assumed the slopes
of the top regression model to be normally distributed with a mean of 0
and a standard deviation of 0.2. This expresses a vague prior knowledge,
allowing both negative and positive slopes of different strength with
equal probability, matching the scarce knowledge basis for the relation
of the different bulk peat chemistry descriptors with the EAC
(supporting figure \ref{fig:p-ppc-res}).\\
For the variance of the individual measurements an exponential
distribution was assumed. For each sample, we estimated a separate
variance. We set the scale parameter for the prior of each measurement
such that an average measurement had an average prior standard deviation
of around \(\SI{138}{\micro\mol\per\g\carbon}\) which corresponds
approximately to the maximum standard error reported by
\citet{Aeschbacher.2012} for EAC measurements for various humic and
fulvic acids using the same measurement procedure. Similarly, for the
residual variance of the regression model an equivalent exponential
distribution was assumed.\\
All regression models were developed and fitted with Stan
\citep{Carpenter.2017} via the R packages rstan (2.19.3)
\citep{StanDevelopmentTeam.2020} and using rstantools (2.0.0)
\citep{Gabry.2019b}. The model was fitted using Markov Chain Monte Carlo
(MCMC) sampling. Four MCMC chains were run for 2000 iterations,
including 500 warmup iterations (6000 post-warmup samples in total).
Model validation was performed using the \(\hat{R}\) statistic, trace
plots of the MCMC draws, posterior predictive checks, area plots of the
estimate regression parameters \citep{Gelman.2014b, Gabry.2019} (R
package bayesplot (1.7.2) \citep{Gabry.2019}), and residual analysis.

\subsection{MIRS-Based Regression Models}

PLSR models were computed using the R package pls (2.7-2)
\citep{Mevik.2019}. Regression models with Bayesian regularization were
computed using rstan (2.19.3) \citep{StanDevelopmentTeam.2020} and
rstanarm (2.19.3) \citep{Goodrich.2020}, and using a vague Gaussian
prior for PLSR-based models and using a horseshoe prior with an assumed
number of relevant variables set to 8 \citep{Piironen.2017c} for
Bayesian regularization models. Bayesian projection was performed using
projpred (1.1.6) \citep{Piironen.2019}. All regression models were
fitted with the same MCMC parameters and validated as described above
for the models based on element ratios.\\
Bayesian projection was performed using projpred (1.1.6)
\citep{Piironen.2019} with L1-search, five clusters for prediction, and
no penalization during computation of projected coefficients
\citep{Piironen.2019, Piironen.2020}. The number of variables to include
in the projected model was determined based on the expected sum of log
predictive densities (ELPD) computed by pareto smoothed importance
sampling (mimicking leave-one-out cross-validation) (PSIS-LOO), such
that the ELPD of the projected model was at most one standard error
smaller than that of the reference model. Even though PSIS-LOO estimates
were biased for most of the models due to influential observations, this
has been shown to have only small influences on the performance of the
projection approach \citep{Piironen.2020}.

\subsection{Cross-Validation}

During 10-fold CV, the data is split into ten folds, and the models are
fitted to all combinations of the folds, leaving out one of the ten
folds each time. The left out fold is then used as test data the model's
prediction are compared to by computing the RMSE \citep{Roberts.2017}.
Peat core data typically have a nested structure (samples at different
depths are nested within cores) in terms of their variability. To ensure
independence of the test data, we used stratified random sampling to
account for this structure \citep{Roberts.2017} which resulted in folds
with 3 to 8 samples. We did not account for the fact that nearby sites
are not completely independent since this would have increased sample
size differences between folds. Due to the relative large inter-site
variability, even between nearby sites, we assumed that this effect is
negligible.

\subsection{Software}

All computations were preformed in R (4.0.1) \citep{RCoreTeam.2020}.
Graphics were created using ggplot2 (3.3.2) \citep{Wickham.2016},
cowplot (1.0.0) \citep{Wilke.2019}, ggrepel (0.8.2)
\citep{Slowikowski.2020}, and ggforce (0.3.1) \citep{Pedersen.2019}, and
Inkscape. Spatial data was handled using sf (0.9-3)
\citep{Pebesma.2018b}, sp (1.4-2) \citep{Pebesma.2005}, rgeos (0.5-3)
\citep{Bivand.2020}, and raster (3.3-13) \citep{Hijmans.2020}. dplyr
(1.0.0) \citep{Wickham.2020e}, purrr (0.3.4) \citep{Henry.2020},
magrittr (1.5) \citep{Bache.2014}, and tidyr (1.1.0)
\citep{Wickham.2020f} were used for general data processing. Measurement
errors and units were handled with quantities (0.1.5)
\citep{Pebesma.2016, Ucar.2019}, and element content data with elco
(0.0.0.9000) \citep{Teickner.2020e}.

\section{Supplementary Results}

\subsection{\texorpdfstring{Correlation of the EAC\textsubscript{POM},
EDC\textsubscript{POM}, and
EDC\textsubscript{POM}:EAC\textsubscript{POM} Ratio with XRF
Data}{Correlation of the EAC, EDC, and EDC:EAC Ratio with XRF Data}}

We measured element contents with wavelength dispersive X-ray
fluorescence spectroscopy (WD-XRF; ZSX Primus II, Rigaku, Tokyo, Japan),
but did not include the measurements into our analyses since such
analyses are entirely explorative and not corrobated by similar analyses
for DOM and HS. Here, we present Pearson correlation coefficients for
the measured element contents with the EAC\textsubscript{POM},
EDC\textsubscript{POM}, and
EDC\textsubscript{POM}:EAC\textsubscript{POM} ratio (table
\ref{tab:t-cor-el-xrf}).

\begin{table}

\caption{\label{tab:t-cor-el-xrf}Pearson correlation between the EAC\textsubscript{POM}, EDC\textsubscript{POM}, and EDC\textsubscript{POM}:EAC\textsubscript{POM} ratio on the one side and element contents as determined by wavelength dispersive X-ray fluorescence spectroscopy.}
\centering
\begin{tabular}[t]{cccc}
\toprule
Element & EAC\textsubscript{POM} & EDC\textsubscript{POM} & EAC\textsubscript{POM}:EDC\textsubscript{POM}\\
\midrule
Na & -0.17 & -0.25 & -0.22\\
Mg & 0.15 & 0.01 & -0.13\\
Al & -0.31 & -0.45 & -0.28\\
Si & -0.25 & -0.40 & -0.24\\
P & -0.08 & -0.51 & -0.52\\
\addlinespace
Cl & -0.06 & 0.00 & -0.10\\
K & -0.31 & -0.55 & -0.35\\
Ca & 0.34 & -0.03 & -0.26\\
Ti & -0.25 & -0.38 & -0.20\\
Cr & -0.32 & -0.43 & -0.26\\
\addlinespace
Mn & 0.22 & -0.15 & -0.33\\
Fe & -0.29 & -0.61 & -0.42\\
Cu & 0.02 & -0.29 & -0.30\\
Zn & 0.23 & -0.03 & -0.14\\
As & 0.07 & -0.01 & -0.02\\
\addlinespace
Br & 0.22 & 0.27 & 0.00\\
Rb & 0.59 & 0.42 & 0.03\\
Sr & 0.23 & -0.07 & -0.31\\
Ba & -0.32 & -0.41 & -0.17\\
Pb & 0.08 & 0.00 & -0.02\\
\bottomrule
\end{tabular}
\end{table}

\section{Supplementary Discussion}

\subsection{\texorpdfstring{Other Potential Mechanisms how Decomposition
may Affect Peat EAC\textsubscript{POM} and
EDC\textsubscript{POM}}{Other Potential Mechanisms how Decomposition may Affect Peat EAC and EDC}}

Different decomposition and other processes may further modify peat
EAC\textsubscript{POM} and EDC\textsubscript{POM}. This may partly
explain the observed variability in the relation of the
EAC\textsubscript{POM} and EDC\textsubscript{POM} to peat chemistry and
molecular structures. However, effects are yet not well explored and and
we can therefore only provide hypotheses.\\
First, it is known that phenols and quinones can undergo different
condensation reactions with other phenols/quinones
\citep{Hotta.2002, Johnson.2015, Bolton.2018, Zhao.2020} or nitrogen and
sulfur containing functional groups
\citep{Bolton.2018, Olk.2006, Heitmann.2006, Yu.2016}. Such reactions
may decrease the amount of phenols and quinones and hence decrease the
EAC\textsubscript{POM} and EDC\textsubscript{POM}. Strong aerobic
decomposition, intesified e.g.~by higher temperatures, higher pH values,
a larger nutrient availability, and a larger fraction of labile OM, may
promote such reactions because it can increase the DOM concentration
\citep{Bragazza.2007, Fenner.2011, Kang.2018c, Bowring.2020} which in
turn is suggested to increase the occurrence of such condensation
reactions \citep{Hotta.2002, Johnson.2015}. However, we may not exclude
that such reactions may also play a role under anoxic conditions. The
negative relation of the EAC\textsubscript{POM} to polysubstituted
aromatics (table \ref{main-tab:t-mir-sel}) may point towards such
effects.\\
Second, intense aerobic decomposition may promote the disintegration of
cell-wall structures \citep{Tsuneda.2001}. This may increase the surface
accessibility of polymeric phenols and quinones to enzymes that can
break them down \citep{Tsuneda.2001}. In addition, this may increase the
frequency of the mentioned condensation reactions. Disintegration of
cell-wall structures may therefore be an important process in the
mineralization and transformation of polymeric phenols and quinones. The
relation of the EAC\textsubscript{POM} and EDC\textsubscript{POM} to
free OH groups and intramolecularly bonded OH groups, respectively, may
be an indicator for this (table \ref{main-tab:t-mir-sel}).\\
Third, polysaccharides are typically mineralized at a faster rate than
polymeric phenols and quinones, both under oxic and anoxic conditions
\citep{Benner.1984}. Since polysaccharides are not assumed to be
redox-active, this may increase the EAC\textsubscript{POM} and
EDC\textsubscript{POM} by residual enrichment of redox active
moieties.\\
Finally, an additional factor that might increase the variability in
relations between peat chemistry and the EAC\textsubscript{POM} and
EDC\textsubscript{POM} is root ingrowth in (highly) decomposed peat.
Roots that are relatively rich in polymeric phenols
\citep{Moore.2007, Scheffer.2000} may increase the
EDC\textsubscript{POM} and EAC\textsubscript{POM} (under anoxic
conditions predominantly the EDC\textsubscript{POM}) of (highly)
decomposed peat. This may result in a relative large
EAC\textsubscript{POM} and/or EDC\textsubscript{POM}, in spite of the
bulk peat being (highly decomposed). This may explain the relative large
EDC\textsubscript{POM}:EAC\textsubscript{POM} ratio for Touxi and
Dongtu. Macrofossil data would have been helpful to answer this
questions, but were not available for the samples under study.\\
Based on our results, it remains currently unclear if and to which
extent these factors may change peat EAC\textsubscript{POM} and
EDC\textsubscript{POM}, but they represent plausible mechanisms which
may explain the variability to peat chemical properties, including
molecular structures.

\subsection{Predictive Performance of the Regression Models}

Several factors may cause the large variability in the predictions of
both models for the EAC\textsubscript{POM} and the bias of the element
ratio-based model. We attribute the latter to the complex interactions
between vegetation chemistry and decomposition, as described above, that
results in non-trivial gradients across the H:C-O:C gradient. The
MIRS-based model probably performs better because it can separate OM
fractions representative for changes in the EAC\textsubscript{POM}. In
addition, element ratios probably cannot fully resolve quinone
structural changes that control peat EAC\textsubscript{POM}, whereas
MIRS-based models can because MIRS allow to better separate variations
in carbonyl groups and different OM fractions \citep{Stuart.2005}. The
large residual uncertainties of both models may be due to the relatively
small sample size in contrast to the broad gradient in peat chemical
properties covered. On the one side, this results in large differences
in peat chemistry the model has to describe, which lowers its predictive
performance in comparison to more targeted models, and one the other
side, the small sample size results in a relative large CV error.
Finally, the electrochemical measurements itself have a relative large
variability which we fully considered during CV. We therefore suggest
that MIRS-based regression models may be at least suitable as screening
tools. Furthermore, we suggest that additional data, different
preprocessing methods, or modeling approaches (e.g.~local models
depending on peat characteristics) may improve the predictive
performance of a predictive model.\\
We believe that the reason for the low performance of the models for the
EDC\textsubscript{POM} is the stronger dependence of phenols on
decomposition, in addition to the general issues mentioned for models
for the EAC\textsubscript{POM}. MIR variables indicative for phenols
\citep{Stuart.2005} typically are overlapped by other bands in peat
MIRS. The stronger dependence on decomposition indicators (e.g.~the N:C
ratio) probably makes the amount of carbonyl groups or aromatic backbone
structures less suitable as predictors for the EDC\textsubscript{POM},
even though the correlation spectra indicated some relation (figure
\ref{fig:p-mir-cor-res}). Similarly, indicators for fresh peat do not
account for the fact that peat with large amounts of polysaccharides
typically contain less polymeric phenols and therefore also may have a
small EDC\textsubscript{POM}. Overall this likely makes predictions
using linear methods or a global model difficult.\\
Bias from not considered contributions of iron to the
EAC\textsubscript{POM} and EDC\textsubscript{POM} may also play a role,
even though residual plots versus total iron content did not indicate
patterns for the filtered data (supporting figures \ref{fig:reg-fe-res}
and \ref{fig:cal-fe-res}).

\clearpage

\section{Supplementary Figures}

Figure \ref{fig:el-preprocessing-p1-res} shows boxplots of the peat
EAC\(_\text{POM}\) and EDC\(_\text{POM}\) replicate measurements for
each sample.

\begin{figure}[H]

{\centering \includegraphics[width=0.5\linewidth]{006-paper-supplementary_files/figure-latex/el-preprocessing-p1-res-1} 

}

\caption{Boxplots of the EAC$_\text{POM}$ and EDC$_\text{POM}$ replicate measurements for the samples from this study. For sample 2, the EAC measurement with more than $\SI{1500}{\micro\mol\per\gram\carbon}$ was assumed to represent a measurement error and therefore discarded prior the statistical analysis.}\label{fig:el-preprocessing-p1-res}
\end{figure}

\clearpage

Figure \ref{fig:p-fe-xrf-comparison-res} compares iron contents from
acid extraction with total iron contents as measured with wavelength
dispersive X-ray fluorescence spectroscopy.

\begin{figure}[H]

{\centering \includegraphics[width=0.5\linewidth]{006-paper-supplementary_files/figure-latex/p-fe-xrf-comparison-res-1} 

}

\caption{Scatterplot of total iron contents from acid extraction versus total iron contents from wavelength dispersive X-ray fluorescence spectroscopy. The diagonal line represents identical iron contents from both procedures.}\label{fig:p-fe-xrf-comparison-res}
\end{figure}

Figure \ref{fig:p-fe-contribution-abs-res} shows histograms of the
calculated contributions of iron to the EAC and EDC and the potential
contribution of iron to either the EAC or EDC. All iron quantities refer
to iron from acid extracts of the samples. It is assumed that
Fe\(^{3+}\) ions contribute one mol electrons to the EAC and that
Fe\(^{2+}\) ions contribute one mol electrons to the EDC. However, the
acid extraction of iron from organic rich samples results in redox
equilibria between the OM and iron species which leads to an
overestimation of Fe\(^{2+}\) over Fe\(^{3+}\). Therefore, the potential
contribution gives the amount of electrons iron may contribute to the
EAC or EDC if all iron was present as Fe\(^{3+}\) or Fe\(^{2+}\),
respectively (\(\text{Fe}_\text{tot}=\text{Fe}^{3+} + \text{Fe}^{2+}\)).
This gives an overview on what contribution iron may maximally have had
to the EAC and EDC, assuming that the acid extraction enabled the
quantification of all redox active iron moieties in the samples.

\begin{figure}[H]

{\centering \includegraphics[width=0.8\linewidth]{006-paper-supplementary_files/figure-latex/p-fe-contribution-abs-res-1} 

}

\caption{Histograms of the calculated contribution of iron to the EAC (left), the EDC (middle), or the potential contribution of iron to either the EAC or EDC (right) to each sample.}\label{fig:p-fe-contribution-abs-res}
\end{figure}

\clearpage

Figure \ref{fig:p-fe-corrected-uncorrected-res} shows plots of EAC and
EDC values with potential contribution from total acid extracted iron
subtracted in dependency of either the measured EAC or EDC or the
EAC\textsubscript{POM} and EDC\textsubscript{POM} (EAC or EDC with
contributions from Fe\(^{3+}\) and Fe\(^{2+}\), respectively,
subtracted) for both the complete and filtered (potential contribution
of iron to the EAC and EDC \(\le \SI{100}{\micro\mol\of\gram\carbon}\)).

\begin{figure}[H]

{\centering \includegraphics[width=1\linewidth]{006-paper-supplementary_files/figure-latex/p-fe-corrected-uncorrected-res-1} 

}

\caption{Plot of EAC or EDC values, where the potential contribution from iron had been subtracted ($\text{EAC} - \text{Fe}_\text{tot}$ or $\text{EDC} - \text{Fe}_\text{tot}$) versus $\text{EAC}_\text{POM}$ or $\text{EDC}_\text{POM}$ values (\textbf{A, B}; $\text{EAC}_\text{POM} = \text{EAC} - \text{Fe}^{3+}$ and $\text{EDC}_\text{POM} = \text{EDC} - \text{Fe}^{2+}$) or measured EAC or EDC values (\textbf{C, D}) for the complete data set (\textbf{A, C}) or the data set with only samples that have a maximum potential contribution of iron to the EAC or EDC $\le \SI{100}{\micro\mol\of\gram\carbon}$ (\textbf{B, D}). Different colours represent samples from different sites and error bars represent standard deviations computed from replicate measurements of the EAC and EDC, respectively. The grey diagonal line with intercept in $(0,0)$ represents identical values along both axes. The other grey line is a linear regression line fitted to the data points.}\label{fig:p-fe-corrected-uncorrected-res}
\end{figure}

\clearpage

Figure \ref{fig:p-fe-raw-iron} shows the relation between measured EAC
and EDC values on the one side and iron content values from wavelength
dispersive X-ray fluorescence spectroscopy measurements and acid
extraction. No systematic relation between the EAC values and iron
contents are visible. For the EDC, samples with higher iron content tend
to have smaller measured EDC values.

\begin{figure}[H]

{\centering \includegraphics[width=1\linewidth]{006-paper-supplementary_files/figure-latex/p-fe-raw-iron-1} 

}

\caption{Scatterplots of the measured EAC and EDC versus iron contents measured with wavelength dispersive X-ray fluorescence spectroscopy (Fe\textsubscript{tot, XRF} in mass-\%) or from acid extraction (Fe$^{2+}$, Fe$^{3+}$, and Fe\textsubscript{tot} in \si{\micro\mol\per\gram\of\sample}). Error bars represent the standard deviation from replicate measurements for the EAC and EDC, respectively.}\label{fig:p-fe-raw-iron}
\end{figure}

\clearpage

Figure \ref{fig:p-fe-decomposition-res} shows iron content values versus
decomposition indicators (N:C ratio and HI\textsubscript{1630/1090}).

\begin{figure}[H]

{\centering \includegraphics[width=0.55\linewidth]{006-paper-supplementary_files/figure-latex/p-fe-decomposition-res-1} 

}

\caption{Scatterplots of iron contents measured with wavelength dispersive X-ray fluorescence spectroscopy (Fe\textsubscript{tot, XRF} in mass-\%) or from acid extraction (Fe$^{2+}$, Fe$^{3+}$, and Fe\textsubscript{tot} in \si{\micro\mol\per\gram\of\sample}) in comparison to peat decomposition indicators (N:C ratio and HI\textsubscript{1630/1090}). The grey lines are regression lines fit to the samples.}\label{fig:p-fe-decomposition-res}
\end{figure}

\clearpage

Figure \ref{fig:p-reg-distogram-res} shows the structure of the Bayesian
hierarchical regression models used to predict the peat
EAC\textsubscript{POM} and EDC\textsubscript{POM} from element ratios,
including the respective distributions
\citep{Kruschke.2015, Baath.2018}, using either Gaussian slopes or
Student-t slopes.

\begin{figure}[H]

{\centering \includegraphics[width=\textwidth]{./../figures/rp_regression_distogram_tot} 

}

\caption{Conceptual representation of the structure of the regression models \citep{Kruschke.2015, Baath.2018} using element ratios as predictor variables. A: Model structure for the models with normal distributed slopes. B: Model structure for the models with Student-t distributed slopes ($\nu = 6$). $y_i$ is the ith replicate EAC\textsubscript{POM} or EDC\textsubscript{POM} measurement. Replicate measurements for the same sample are assumed to follow a Gamma distribution with mean $\mu_j$ for the jth sample, and a sample-specific standard deviation $\sigma_j$. The $\mu_j$ are modeled by a regression equation $\exp\left(\beta_0 + \beta_1 x_1\right)$ and are also assumed to follow a Gamma distribution with mean $\mu$ and standard deviation $\sigma_2$. $\beta_0$ is the global intercept of the regression equation and is assumed to follow a normal distribution with mean $\mu_{\beta_0}$ and standard deviation $\sigma_{\beta_0}$. $\beta_1$ represents a regression slope for a variable $x_1$ (e.g. the H:C ratio). For each of the $k$ predictor variables ($k=2$ for the models with the H:C and O:C ratio as predictor variables, and $k=4$ for the models including in addition the N:C and S:C ratio), there is a seprate $\beta_k$ (not shown in the figure). The $\beta_k$ (e.g. $\beta_1$) are either assumed to follow a normal distribution (A), or a Student-t distribution (B) with mean $\mu_{\beta_1}$, standard deviation $\sigma_{\beta_1}$, and $\nu_{\beta_1}$ degrees of freedom.}\label{fig:p-reg-distogram-res}
\end{figure}

\clearpage

Figure \ref{fig:p-ppc-res} shows the prior predictive check for the
slopes of potential predictor variables for the different modeling
approaches for the element ratio-based models.

\begin{figure}[H]

{\centering \includegraphics[width=1\linewidth]{006-paper-supplementary_files/figure-latex/p-ppc-res-1} 

}

\caption{Prior predictive checks for the regression model using element ratio as predictor variables. Columns differentiate different modeling approaches (Gaussian or Student-t prior for slopes, all element ratios as preditor variables or only the H:C and O:C ratio), and rows represent different predictor variables. Lines are 100 random draws from the respective prior distributions.}\label{fig:p-ppc-res}
\end{figure}

\clearpage

Figure \ref{fig:el-comparison-peatland-types-res} shows the peat
samples' EAC\textsubscript{POM} and EDC\textsubscript{POM} values of all
peat samples for each peatland site.

\begin{figure}[H]

{\centering \includegraphics[width=0.7\linewidth]{006-paper-supplementary_files/figure-latex/el-comparison-peatland-types-res-1} 

}

\caption{Plots of the EAC$_\text{POM}$ and EDC$_\text{POM}$ (average values of replicate measurements) for each peatland site and assigned to different peatland types following information from available studies for the respective sites (table ef{tab:t-study-sites}) or from own investigations following concepts in \citet{Rydin.2013}.}\label{fig:el-comparison-peatland-types-res}
\end{figure}

\clearpage

Figure \ref{fig:reg-fe-res} shows scatterplots of the
EAC\textsubscript{POM}, EDC\textsubscript{POM}, and
EAC\textsubscript{POM}:EDC\textsubscript{POM} ratio versus several peat
chemistry indicators.

\begin{figure}[H]

{\centering \includegraphics[width=\textwidth]{006-paper-supplementary_files/figure-latex/p-el-variables1-res-1} 

}

\caption{\textbf{a:} Scatterplots of the EAC\textsubscript{POM}, EDC\textsubscript{POM}, and EDC\textsubscript{POM}:EAC\textsubscript{POM} ratio versus various peat properties. Element ratios are molar ratios. "HI" is the ratio of the intensities at 1630 and \SI{1090}{\wn} from mid infrared spectra (larger values indicate more decomposed peat). C$_\text{OX}$ is the nominal oxidation state of carbon \citep{Masiello.2008}. Only samples with a potential contribution of iron to the EAC and EDC $\le\SI{100}{\micro\mol\per\gram\carbon}$ are shown in all plots.}\label{fig:p-el-variables1-res}
\end{figure}

\clearpage

Figure \ref{fig:reg-fe-res} shows Pearson correlation spectra for the
EAC\textsubscript{POM} and EDC\textsubscript{POM} using the MIRS for the
peat samples.

\begin{figure}[H]

{\centering \includegraphics[width=\textwidth]{006-paper-supplementary_files/figure-latex/p-mir-cor-res-1} 

}

\caption{Pearson correlation spectra for the EAC\textsubscript{POM} and EDC\textsubscript{POM}, respectively. Each point on the lines represents the Pearson correlation coefficient at the respective wavenumber. The uppermost grey line is an arbitrary reference MIR spectrum from the data set plotted to facilitate interpretation. Correlation coefficients were computed for samples with a potential contribution of iron to the EAC and EDC $\le\SI{100}{\micro\mol\per\gram\carbon}$.}\label{fig:p-mir-cor-res}
\end{figure}

\clearpage

Figure \ref{fig:reg-fe-res} shows scatterplots of the residuals of the
element ratio-based regression models in dependency of the total iron
content of the samples.

\begin{figure}[H]

{\centering \includegraphics[width=1\linewidth]{006-paper-supplementary_files/figure-latex/reg-fe-res-1} 

}

\caption{Plot of residuals versus total iron contents for the element ratio-based regression models with the complete data set (\textbf{A}) and the filtered data set (\textbf{B}, maximal potential contribution of iron to the EAC and EDC $\le\SI{100}{\micro\mol\per\gram\carbon}$). Columns contain plots for different modeling approaches and dependent variables (Gaussian or Student-t prior for slopes, EAC\textsubscript{POM} or EDC\textsubscript{POM} as dependent variable), and rows represent different predictor variable combinations (only H:C and O:C ratio or all element ratios). The red curves are LOESS smoothers fit to the samples.}\label{fig:reg-fe-res}
\end{figure}

\clearpage

Figure \ref{fig:cal-fe-res} shows scatterplots of the residuals of the
MIRS-based regression models in dependency of the total iron content of
the samples.

\begin{figure}[H]

{\centering \includegraphics[width=0.8\linewidth]{006-paper-supplementary_files/figure-latex/cal-fe-res-1} 

}

\caption{Plot of residuals versus total iron contents for the MIRS-based regression models with the complete data set (\textbf{A}) and the filtered data set (\textbf{B}, maximal potential contribution of iron to the EAC and EDC $\le\SI{100}{\micro\mol\per\gram\carbon}$). Columns contain plots for different dependent variables (EAC\textsubscript{POM} or EDC\textsubscript{POM}), and rows represent different modeling approaches (Bayesian regularization or PLSR as methods to compute regression models, using the non derived or first derivative of the mid infrared spectra as predictor variables, transforming the dependent variable (EDC\textsubscript{POM} only) by dividing it by HI$_\text{1630/1090}$). The red lines are LOESS smoothers fit to the samples.}\label{fig:cal-fe-res}
\end{figure}

\clearpage

Figure \ref{fig:p-y-yhat-res} shows scatterplots of measured versus
predicted values for the element ratio-based and MIRS-based regression
models for the filtered data set which were interpreted in detail
throughout the main text.

\begin{figure}[H]

{\centering \includegraphics[width=0.6\textwidth]{006-paper-supplementary_files/figure-latex/p-y-yhat-res-1} 

}

\caption{Plot of measured values versus fitted values for the regression models predicting the EAC\textsubscript{POM} and EDC\textsubscript{POM} using the filtered data set. Rows contain plots for the two different modeling approaches: "Element ratios" indicates the regression models using all four element ratios (H:C, O:C, N:S, S:C ratio) as predictor variables, and "MIRS" the full regression models using non-derived mid infrared spectra. Error bars represent 95\% predictive intervals for the measured and predicted values. Points on the diagonal lines represents identical measured and fitted values.}\label{fig:p-y-yhat-res}
\end{figure}

\clearpage

Figure \ref{fig:p-y-yhat-reg-res} shows plots of measured versus fitted
values for all computed element ratio-based regression models.

\begin{figure}[H]

{\centering \includegraphics[width=1\linewidth]{006-paper-supplementary_files/figure-latex/p-y-yhat-reg-res-1} 

}

\caption{Plot of measured versus fitted values (both in \si{\micro\mol\per\gram\carbon}) for the element ratio-based regression models with the complete data set (\textbf{A}) and the filtered data set (\textbf{B}, maximal potential contribution of iron to the EAC and EDC $\le\SI{100}{\micro\mol\per\gram\carbon}$). Columns contain plots for different modeling approaches and dependent variables (Gaussian or Student-t prior for slopes, EAC\textsubscript{POM} or EDC\textsubscript{POM} as dependent variable), and rows represent different predictor variable combinations (only H:C and O:C ratio or all element ratios). Error bars represent 95\% predictive intervals for the measured and predicted values. Points on the diagonal lines represents identical measured and fitted values.}\label{fig:p-y-yhat-reg-res}
\end{figure}

\clearpage

Figure \ref{fig:p-y-yhat-cal-res} shows plots of measured versus fitted
values for all computed MIRS-based regression models.

\begin{figure}[H]

{\centering \includegraphics[width=0.8\linewidth]{006-paper-supplementary_files/figure-latex/p-y-yhat-cal-res-1} 

}

\caption{Plot of measured versus fitted values (both in \si{\micro\mol\per\gram\carbon}) for the MIRS-based regression models with the complete data set (\textbf{A}) and the filtered data set (\textbf{B}, maximal potential contribution of iron to the EAC and EDC $\le\SI{100}{\micro\mol\per\gram\carbon}$). Columns contain plots for different dependent variables (EAC\textsubscript{POM} or EDC\textsubscript{POM}), and rows represent different modeling approaches (Bayesian regularization or PLSR as methods to compute regression models, using the non derived or first derivative of the mid infrared spectra as predictor variables, transforming the dependent variable (EDC only) by dividing it by HI$_\text{1630/1090}$). Error bars represent 95\% predictive intervals for the measured and predicted values. Points on the diagonal lines represents identical measured and fitted values.}\label{fig:p-y-yhat-cal-res}
\end{figure}

\clearpage

Figure \ref{fig:p-cal-elpd-res} shows plots of the expected sum of log
predictive densities (ELPD) estimated by PSIS-LOO
\citep{Piironen.2020, Piironen.2019} for the MIRS-based regression
models (reference models) and the projected models.

\begin{figure}[H]

{\centering \includegraphics[width=0.75\textwidth]{006-paper-supplementary_files/figure-latex/p-cal-elpd-res-1} 

}

\caption{Plot of the expected sum of log predictive densities (ELPD) for the MIRS-based regression models (reference models) and the projected models with the complete data set (top row) and the filtered data set (bottom row, maximal potential contribution of iron to the EAC and EDC $\le\SI{100}{\micro\mol\per\gram\carbon}$). Points reresent the average ELPD, and error bars its standard error. The y axis differentiates modeling approaches (Bayesian regularization or PLSR as regression model, not derived or first derivative of spectra as predictor variables, transformation of the EDC by division by HI$_\text{1630/1090}$). Note that the ELPD values are biased since not for all models the underlying assumptions were met.}\label{fig:p-cal-elpd-res}
\end{figure}

\clearpage

Figure \ref{fig:p-partial-dependence-res} shows partial dependence plots
for groups of variables in the projected models for the MIRS-based
regression models. Variables were assigned to different OM groups based
on the molecular structures the MIRS variables can be assigned to.

\begin{figure}[H]

{\centering \includegraphics[width=0.8\linewidth]{006-paper-supplementary_files/figure-latex/p-partial-dependence-res-1} 

}

\caption{Partial dependence plots for groups of variables included in the projected MIRS-based regression model for the EAC\textsubscript{POM} (with Bayesian regularization and not derived spectra) computed with the complete data set (\textbf{A}) and the filtered data set (\textbf{B}, maximal potential contribution of iron to the EAC and EDC $\le\SI{100}{\micro\mol\per\gram\carbon}$). This is the model interpreted in the main text. Each panel contains the plots for one OM group (labile, lipids, aromatics, carbonyl), and each panel's subplots the respective variables indicated by their wavenumber values (for example MIR variables assigned to labile OM fractions are those at 3660 and \SI{3670}{\wn}). The partial dependence plots were created by predicting EAC\textsubscript{POM} values with the projected models for the input data, where all variables except those in the respective OM groups were set to their average value. Black points represent these predictions. Grey points represent the respective measured EAC\textsubscript{POM}. Error bars represent the 95\%-posterior intervals for the average (predicted values) and individual observations (for the measured values), respectively.}\label{fig:p-partial-dependence-res}
\end{figure}

\clearpage

Figure \ref{fig:p-el-eac-edc2-res} shows plots of the EDC versus the EAC
for samples from this study and different HS and DOM samples from other
studies.

\begin{figure}[H]

{\centering \includegraphics[width=0.7\textwidth]{006-paper-supplementary_files/figure-latex/p-el-eac-edc2-res-1} 

}

\caption{Plot of the average EAC versus the average EDC for peat POM samples of this study and DOM and HS samples from other studies. Error bars represent the respective standard errors from replicate measurements (only for this study). Samples below the diagonal line have a larger EAC than EDC.}\label{fig:p-el-eac-edc2-res}
\end{figure}

\clearpage

Figure \ref{fig:conceptual2} is a conceptual representation of the
interactions between vegetation chemistry and decomposition on peat POM
electrochemical properties as mediated by its chemistry.

\begin{figure}[H]

{\centering \includegraphics[width=1\linewidth]{./../figures/conceptual_vegetation_decomposition2} 

}

\caption{Conceptual graphic showing the assumed effects of vegetation chemistry (polymeric phenol content) and decomposition pathways and intensity on peat chemistry and its EAC\textsubscript{POM} and EDC\textsubscript{POM}. Circles represent from left to right phenols, quinones, polysaccharides, and non-redox active OM fractions (condensed aromatics, lipids). Their area represents the absolute fraction of these moieties. Barplots represent the absolute EDC\textsubscript{POM} and EAC\textsubscript{POM}.}\label{fig:conceptual2}
\end{figure}

\bibliography{references.bib}


\end{document}
